% LaTeX source for textbook ``How to think like a computer scientist''
% Copyright (C) 1999  Allen B. Downey
% Copyright (C) 2009  Thomas Scheffler


\documentclass[a4paper]{book}
\usepackage[ngerman,english]{babel}
\usepackage[latin1]{inputenc}
\usepackage[T1]{fontenc}
\usepackage{lmodern}
\usepackage{epsfig}
\usepackage{makeidx}
\usepackage{url}
\usepackage{fancyhdr}
\usepackage{multicol}
\usepackage[hidelinks]{hyperref}
\usepackage{longtable}
\usepackage{ifthen}
\usepackage{boxedminipage}
\usepackage{todonotes}

% conditional compilation of the document for different languages

%\newboolean{German}
%\setboolean{German}{false}

% the exercise environment

\newcounter{exercisenum}                                                  
     
% by default, the exercise number includes the chapter number             
% this way, an exercise label is a complete, unique exercise id           
     
\renewcommand{\theexercisenum}{{\thechapter}.\arabic{exercisenum}}  

% Standard font size for exercise/problem text                            

\newenvironment{exercisesize}{\begin{small}}{\end{small}}                 

\newcommand{\exerciseheader}[2]{                                          
     
  \begin{exercisesize}                                                    
     
  % Use alphabetic chars for subparts of exercises,                            
  % and roman numerals for subparts of them.
     
  \def\theenumi{\alph{enumi}}                                             
  \def\labelenumi{\theenumi.}                                             
  \def\theenumii{\roman{enumii}}                                          
  \def\labelenumii{\theenumii.}                                           
  {\bf Exercise {#1}{#2}}\hspace{0.1in}                 
}                                                                         

\newcommand{\startexercise}[1]{%
  \refstepcounter{exercisenum}                                            
  \exerciseheader{\theexercisenum}{#1}                                    
}                                                                         

\newcommand{\stopexercise}{%                                                   
  {\hfill}                                                               
  \end{exercisesize}      
}                                                         
     
\newcommand{\normaldif}{}                                                 
     
\newcommand{\bigdif}{\dag{}}                                              
     
\newcommand{\verybigdif}{\ddag{}}             

\newenvironment{exercise}{\startexercise{\normaldif{}}}{\stopexercise}    
     
\newenvironment{hardexercise}{\startexercise{\bigdif{}}}{\stopexercise}   
     
%% end of the exercise environment


%%------------------------------------------------------------
% formatting commands

\sloppy
\setlength{\topmargin}{0.125in}
\setlength{\oddsidemargin}{0.875in}
\setlength{\evensidemargin}{0.875in}

\setlength{\headsep}{3ex}
\setlength{\textheight}{8in}

\setlength{\parindent}{0.0in}
\setlength{\parskip}{1.7ex plus 0.5ex minus 0.5ex}
\renewcommand{\baselinestretch}{1.02}

% see LaTeX Companion page 62
\setlength{\topsep}{-0.0\parskip}
\setlength{\partopsep}{-0.5\parskip}
\setlength{\itemindent}{0.0in}
\setlength{\listparindent}{0.0in}

% see LaTeX Companion page 26
% these are copied from /usr/local/teTeX/share/texmf/tex/latex/base/book.cls
% all I changed is afterskip

\makeatletter
\renewcommand{\section}{\@startsection 
    {section} {1} {0mm}%
    {-3.5ex \@plus -1ex \@minus -.2ex}%
    {0.7ex \@plus.2ex}%
    {\normalfont\Large\bfseries}}
\renewcommand\subsection{\@startsection {subsection}{2}{0mm}%
    {-3.25ex\@plus -1ex \@minus -.2ex}%
    {0.3ex \@plus .2ex}%
    {\normalfont\large\bfseries}}
\renewcommand\subsubsection{\@startsection {subsubsection}{3}{0mm}%
    {-3.25ex\@plus -1ex \@minus -.2ex}%
    {0.3ex \@plus .2ex}%
    {\normalfont\normalsize\bfseries}}

\makeatother

\newcommand{\beforeverb}{\vspace{0.6\parskip}}
\newcommand{\afterverb}{\vspace{0.6\parskip}}

\newcommand{\adjustpage}[1]{\enlargethispage{#1\baselineskip}}
\newcommand{\clearemptydoublepage}{\newpage{\pagestyle{empty}\cleardoublepage}}
\newcommand{\blankpage}{\pagestyle{empty}\vspace*{1in}\newpage}

\newcommand{\beforefig}{\vspace{1.3\parskip}}
\newcommand{\afterfig}{\vspace{-0.2\parskip}}
\newcommand{\myfig}[1]{
    \beforefig
    \centerline{\epsfig{#1,scale=0.8}}
    \afterfig
}

\newcommand{\beforechapter}{
%    \clearemptydoublepage 
    \cleardoublepage 
    \setcounter{exercisenum}{0}
}

\pagestyle{fancyplain}

\renewcommand{\chaptermark}[1]{\markboth{#1}{}}
\renewcommand{\sectionmark}[1]{\markright{\thesection\ #1}{}}

\lhead[\fancyplain{}{\bfseries\thepage}]%
      {\fancyplain{}{\bfseries\rightmark}}
\rhead[\fancyplain{}{\bfseries\leftmark}]%
      {\fancyplain{}{\bfseries\thepage}}
\cfoot{}

% turn off the rule under the header
%\setlength{\headrulewidth}{0pt}

% the following is a brute-force way to prevent the headers
% from getting transformed into all-caps
\renewcommand\MakeUppercase{}


\sloppy
\setlength{\topmargin}{0.75in}
\setlength{\headsep}{0.5in}
\setlength{\oddsidemargin}{1.0in}
\setlength{\evensidemargin}{.95in}
\makeindex


%%-----------------------------------------------------------
% beginning of the document

\begin{document}

\thispagestyle{empty}

\begin{flushright}
\vspace*{2.5in}

{\huge How to Think Like a Computer Scientist}

\vspace{0.25in}

{\LARGE C Version}

\vspace{1in}

{\Large Thomas Scheffler}

{based on previous work by Allen B. Downey}

\vspace{1in}

{\Large Version 1.10}

{\small June 27th, 2019}
\vfill

\end{flushright}



Copyright (C) 1999  Allen B. Downey\\
Copyright (C) 2009  Thomas Scheffler

\vspace{0.25in}

Permission is granted to copy, distribute, transmit and adapt this
work under the Creative Commons Attribution-NonCommercial-ShareAlike 4.0
International License: \url{https://creativecommons.org/licenses/by-nc/4.0/}.

If you are interested in distributing a commercial version of this
work, please contact the author(s).

The \LaTeX\ source and code for this book is available from: \\
\url{https://github.com/tscheffl/ThinkC}

\frontmatter
\tableofcontents

\mainmatter
% LaTeX source for textbook ``How to think like a computer scientist''
% Copyright (C) 1999  Allen B. Downey
% Copyright (C) 2009  Thomas Scheffler

\chapter{The way of the program}
\label{chap01}

The goal of this book, and this class, is to teach you to think like a
computer scientist.  I like the way computer scientists think because
they combine some of the best features of Mathematics, Engineering,
and Natural Science.  Like mathematicians, computer scientists use formal
languages to denote ideas (specifically computations).  Like
engineers, they design things, assembling components into systems and
evaluating tradeoffs among alternatives.  Like scientists,
they observe the behavior of complex systems, form hypotheses, and test
predictions.

The single most important skill for a computer scientist is {\bf
problem-solving}.  By that I mean the ability to formulate problems,
think creatively about solutions, and express a solution clearly and
accurately.  As it turns out, the process of learning to program is an
excellent opportunity to practice problem-solving skills.  That's why
this chapter is called ``The way of the program.''

On one level, you will be learning to program, which is a useful
skill by itself.  On another level you will use programming
as a means to an end.  As we go along, that end will
become clearer.

\section{What is a programming language?}
\index{programming language}
\index{language!programming}

The programming language you will be learning is C, which was developed
in the early 1970s by Dennis M. Ritchie at the Bell Laboratories.  C is
an example of a {\bf high-level language}; other high-level languages
you might have heard of are Pascal, C++ and Java.

As you might infer from the name ``high-level language,'' there are
also {\bf low-level languages}, sometimes referred to as machine
language or assembly language.  Loosely-speaking, computers can only
execute programs written in low-level languages.  Thus, programs
written in a high-level language have to be translated before they can
run.  This translation takes some time, which is a small disadvantage
of high-level languages.

\index{portable}
\index{high-level language}
\index{low-level language}
\index{language!high-level}
\index{language!low-level}

But the advantages are enormous.  First,
it is {\em much} easier to program in a high-level language;
by ``easier'' I mean that the program takes less time to write,
it's shorter and easier to read, and it's more likely to be
correct.  Secondly, high-level languages are {\bf portable},
meaning that they can run on different kinds of computers with
few or no modifications.  Low-level programs can only run
on one kind of computer, and have to be rewritten to run on
another.

Due to these advantages, almost all programs are written in
high-level languages.  Low-level languages are only used for
a few special applications.

\index{compile}
\index{interpret}

There are two ways to translate a program;
{\bf interpreting} or {\bf compiling}.  An interpreter
is a program that reads a high-level program
and does what it says.  In effect, it translates
the program line-by-line, alternately reading lines and
carrying out commands.

\vskip 0.7em
\centerline{\includegraphics[height=3cm]{figs/Interpreter}}

A compiler is a program that reads a high-level program and
translates it all at once, before executing any of the commands.
Often you compile the program as a separate step, and then
execute the compiled code later.  In this case, the high-level
program is called the {\bf source code}, and the translated
program is called the {\bf object code} or the {\bf executable}.

As an example, suppose you write a program in C.  You might
use a text editor to write the program (a text editor is
a simple word processor).  When the program is finished, you
might save it in a file named {\tt program.c}, where ``program''
is an arbitrary name you make up, and the suffix {\tt .c} is
a convention that indicates that the file contains C source
code.

Then, depending on what your programming environment is like,
you might leave the text editor and run the compiler.  The
compiler would read your source code, translate it, and create
a new file named {\tt program.o} to contain the object code,
or {\tt program.exe} to contain the executable. 

\vskip 0.7em
\centerline{\includegraphics[height=3cm]{figs/Compiler}}


The next step is to run the program, which requires some kind of executor. 
The role of the executor is to load the program (copy it from disk into memory) 
and make the computer start executing the program. 


Although this process may seem complicated, in most programming
environments (sometimes called development environments), these steps
are automated for you.  Usually you will only have to write a program
and press a button or type a single command to compile and run it.  On
the other hand, it is useful to know what the steps are that are
happening in the background, so that if something goes wrong you can
figure out what it is.

% Leftover: when is compilation better than interpretation?

\section{What is a program?}

A program is a sequence of instructions that
specifies how to perform a computation.  The computation might be
something mathematical, like solving a system of equations or finding
the roots of a polynomial, but it can also be a symbolic computation,
like searching and replacing text in a document or (strangely enough)
compiling a program.

\index{statement}

The instructions, which we will call {\bf statements}, look different
in different programming languages, but there are a few basic
operations most languages can perform:

\begin{description}

\item[input:] Get data from the keyboard, or a file, or some
other device.

\item[output:] Display data on the screen or send data to a
file or other device.

\item[math:] Perform basic mathematical operations like addition and
multiplication.

\item[testing:] Check for certain conditions and execute the
appropriate sequence of statements.

\item[repetition:] Perform some action repeatedly, usually with
some variation.

\end{description}

That's pretty much all there is to it.
Every program you've ever used, no matter how complicated, is
made up of statements that perform these operations.  Thus,
one way to describe programming is the process of breaking a
large, complex task up into smaller and smaller subtasks
until eventually the subtasks are simple enough to be performed
with one of these basic operations.

\section{What is debugging?}
\index{debugging}
\index{bug}

Programming is a complex process, and since it is done by
human beings, it often leads to errors.  For whimsical reasons,
programming errors are called {\bf bugs} and the process
of tracking them down and correcting them is called
{\bf debugging}.

There are a few different kinds of errors that can occur
in a program, and it is useful to distinguish between them
in order to track them down more quickly.

\subsection{Compile-time errors}
\index{compile-time error}
\index{error!compile-time}

The compiler can only translate a program if the program is
syntactically correct; otherwise, the compilation fails and
you will not be able to run your program.  {\bf Syntax}
refers to the structure of your program and the rules about
that structure.

\index{syntax}

For example, in English, a sentence must begin with a capital
letter and end with a period.  this sentence contains a syntax
error.  So does this one

For most readers, a few syntax errors are not a significant
problem, which is why we can read the poetry of E.~E.~Cummings
without spewing error messages.

Compilers are not so forgiving.  If there is a single syntax
error anywhere in your program, the compiler will print an
error message and quit, and you will not be able to run
your program.

To make matters worse, there are more syntax rules in C
than there are in English, and the error messages you get from
the compiler are often not very helpful.  During the first
few weeks of your programming career, you will probably
spend a lot of time tracking down syntax errors.  As you
gain experience, though, you will make fewer errors and find
them faster.

\subsection{Run-time errors}
\label{run-time}
\index{run-time error}
\index{error!run-time}
%\index{exception}
\index{safe language}
\index{language!safe}

The second type of error is a run-time error, so-called because
the error does not appear until you run the program.  

C is not a {\bf safe} language, such as Java, where
run-time errors are rare. Programming in C allows you to get very close to the actual
computing hardware. Most run-time errors C occur because the
language provides no protection against the accessing or
overwriting of data in memory.

For the simple sorts of programs we will be writing for the next few weeks, 
run-time errors are rare, so it might be a little while before you encounter one.


\subsection{Logic errors and semantics}
\index{semantics}
\index{logic error}
\index{error!logic}

The third type of error is the {\bf logical} or {\bf semantic}
error.  If there is a logical error in your program, it will
compile and run successfully, in the sense that the computer
will not generate any error messages, but it will not do the
right thing.  It will do something else.  Specifically, it will
do what you told it to do.

The problem is that the program you wrote is not the program
you wanted to write.  The meaning of the program (its semantics)
is wrong.  Identifying logical errors can be tricky, since
it requires you to work backwards by looking at the output
of the program and trying to figure out what it is doing.

\subsection{Experimental debugging}

One of the most important skills you will acquire in this
class is debugging.  Although it can be frustrating, debugging
is one of the most intellectually rich, challenging, and
interesting parts of programming.

In some ways debugging is like detective work.  You are
confronted with clues and you have to infer the processes
and events that lead to the results you see.

Debugging is also like an experimental science.  Once you have an idea
what is going wrong, you modify your program and try again.  If your
hypothesis was correct, then you can predict the result of the
modification, and you take a step closer to a working program.  If
your hypothesis was wrong, you have to come up with a new one.  As
Sherlock Holmes pointed out, ``When you have eliminated the
impossible, whatever remains, however improbable, must be the truth.''
(from A. Conan Doyle's {\em The Sign of Four}).

\index{Holmes, Sherlock}
\index{Doyle, Arthur Conan}

For some people, programming and debugging are the
same thing.  That is, programming is the process of gradually
debugging a program until it does what you want.  The idea
is that you should always start with a working program that
does {\em something}, and make small modifications, debugging
them as you go, so that you always have a working program.

For example, Linux is an operating system that contains thousands of
lines of code, but it started out as a simple program Linus Torvalds
used to explore the Intel 80386 chip.  According to Larry Greenfield,
``One of Linus's earlier projects was a program that would switch
between printing AAAA and BBBB.  This later evolved to Linux''
(from {\em The Linux Users' Guide} Beta Version 1).

\index{Linux}

In later chapters I will make more suggestions about debugging
and other programming practices.

\section{Formal and natural languages}
\index{formal language}
\index{natural language}
\index{language!formal}
\index{language!natural}

{\bf Natural languages} are the languages that people speak,
like English, Spanish, and French.  They were not designed
by people (although people try to impose some order on them);
they evolved naturally.

{\bf Formal languages} are languages that are designed by people for
specific applications.  For example, the notation that mathematicians
use is a formal language that is particularly good at denoting
relationships among numbers and symbols.  Chemists use a formal
language to represent the chemical structure of molecules.  And
most importantly:

\begin{quote}
{\bf Programming languages are formal languages that have been
designed to express computations.}
\end{quote}

As I mentioned before, formal languages tend to have strict rules
about syntax.  For example, $3+3=6$ is a syntactically correct
mathematical statement, but $3=+6\$$ is not.  Also, $H_2O$ is a
syntactically correct chemical name, but $_2Zz$ is not.

Syntax rules come in two flavors, pertaining to tokens and structure.
Tokens are the basic elements of the language, like words and numbers
and chemical elements.  One of the problems with {\tt 3=+6\$} is that
{\tt \$} is not a legal token in mathematics (at least as far as I
know).  Similarly, $_2Zz$ is not legal because there is no element with
the abbreviation $Zz$.

The second type of syntax rule pertains to the structure of a
statement; that is, the way the tokens are arranged.  The statement
{\tt 3=+6\$} is structurally illegal, because you can't have a plus
sign immediately after an equals sign.  Similarly, molecular formulas
have to have subscripts after the element name, not before.

When you read a sentence in English or a statement in a formal
language, you have to figure out what the structure of the sentence is
(although in a natural language you do this unconsciously).  This
process is called {\bf parsing}.

\index{parse}

For example, when you hear the sentence, ``The other shoe fell,'' you
understand that ``the other shoe'' is the subject and ``fell'' is the
verb.  Once you have parsed a sentence, you can figure out what it
means, that is, the semantics of the sentence.  Assuming that you know
what a shoe is, and what it means to fall, you will understand the
general implication of this sentence.

Although formal and natural languages have many features in
common---tokens, structure, syntax and semantics---there are many
differences.

\index{ambiguity}
\index{redundancy}
\index{literalness}

\begin{description}

\item[ambiguity:] Natural languages are full of ambiguity, which
people deal with by using contextual clues and other information.
Formal languages are designed to be nearly or completely unambiguous,
which means that any statement has exactly one meaning,
regardless of context.

\item[redundancy:] In order to make up for ambiguity and reduce
misunderstandings, natural languages employ lots of
redundancy.  As a result, they are often verbose.  Formal languages
are less redundant and more concise.

\item[literalness:] Natural languages are full of idiom and
metaphor.  If I say, ``The other shoe fell,'' there is probably
no shoe and nothing falling.  Formal languages mean
exactly what they say.

\end{description}

People who grow up speaking a natural language (everyone) often have a
hard time adjusting to formal languages.  In some ways the difference
between formal and natural language is like the difference between
poetry and prose, but more so:

\index{poetry}
\index{prose}

\begin{description}

\item[Poetry:] Words are used for their sounds as well as for
their meaning, and the whole poem together creates an effect or
emotional response.  Ambiguity is not only common but often
deliberate.

\item[Prose:] The literal meaning of words is more important
and the structure contributes more meaning.  Prose is more amenable to
analysis than poetry, but still often ambiguous.

\item[Programs:] The meaning of a computer program is unambiguous
and literal, and can be understood entirely by analysis of the
tokens and structure.

\end{description}

Here are some suggestions for reading programs (and other formal
languages).  First, remember that formal languages are much more dense
than natural languages, so it takes longer to read them.  Also, the
structure is very important, so it is usually not a good idea to read
from top to bottom, left to right.  Instead, learn to parse the
program in your head, identifying the tokens and interpreting the
structure.  Finally, remember that the details matter.  Little things
like spelling errors and bad punctuation, which you can get away
with in natural languages, can make a big difference in a formal
language.

\section{The first program}
\label{hello}
\index{hello world}

Traditionally the first program people write in a new language
is called ``Hello, World.'' because all it does is display the
words ``Hello, World.''  In C, this program looks like this:

\begin{verbatim}
  #include <stdio.h>
  #include <stdlib.h>

  /* main: generate some simple output */

  int main(void)
  {
        printf("Hello, World.\n");
        return(EXIT_SUCCESS);
  }

\end{verbatim}
%
Some people judge the quality of a programming language by
the simplicity of the ``Hello, World.'' program.  By this
standard, C does reasonably well.  Even so, this simple program 
contains several features that are hard to explain 
to beginning programmers. For now, we will ignore some of them, 
like the first two lines.

\index{comment}
\index{statement!comment}

The third line begins with {\tt /*} and ends with  {\tt */}, which indicates
that it is a {\bf comment}.  A comment is a bit of
English text that you can put in the middle of a program,
usually to explain what the program does.  When the compiler
sees a {\tt /*}, it ignores everything from there until it finds the corresponding
 {\tt */}.

In the forth line, you notice the word {\tt main}.  {\tt main} is a
special name that indicates the place in the program where execution
begins.  When the program runs, it starts by executing the first
{\bf statement} in {\tt main()} and it continues, in order, until it gets
to the last statement, and then it quits.


\index{printf()}
\index{statement!printf}	

There is no limit to the number of statements that can be in 
{\tt main()}, but the example contains only two. 
The first  is an {\bf output} statement, 
meaning that it displays or prints a message on the screen.
The second statement tells the operating system that our program
executed successfully.  

The statement that prints things on the screen is
{\tt printf()}, and the characters between the quotation marks
will get printed. Notice the {\tt \textbackslash n} after the 
last character. This is a special character called \emph{newline} that is appended at the end 
of a line of text  and causes the cursor to move to the next line of the display. 
The next time you output something, the new text appears on the next line.
At the end of the statement
there is a semicolon ({\tt ;}), which is required at the end
of every statement.

There are a few other things you should notice about the syntax
of this program.  First, C uses curly-brackets (\{ and
\}) to group things together. 
In this case, the output statement
is enclosed in curly-brackets, indicating that it is {\em inside} the
definition of {\tt main()}.  Also, notice that the statement is
indented, which helps to show visually which lines are inside the
definition.

At this point it would be a good idea to sit down in front of
a computer and compile and run this program.  The details of how
to do that depend on your programming environment, this book 
assumes that you know how to do it.

As I mentioned, the C compiler is very pedantic with syntax.
If you make any errors when you type in the program, chances
are that it will not compile successfully.  For example, if
you misspell {\tt stdio.h}, you might get an error message like
the following:

\begin{verbatim}
   hello_world.c:1:19: error: sdtio.h: No such file or directory
\end{verbatim}
%
There is a lot of information on this line, but it is presented
in a dense format that is not easy to interpret.  A more friendly
compiler might say something like:

\begin{quote}
``On line 1 of the source code file named hello\_world.c, you tried to
include a header file named sdtio.h.  I didn't find anything
with that name, but I did find something named stdio.h.  Is
that what you meant, by any chance?''
\end{quote}

Unfortunately, few compilers are so accommodating.  The compiler
is not really very smart, and in most cases the error message
you get will be only a hint about what is wrong.  It will take
some time for you to learn to interpret different compiler messages.

Nevertheless, the compiler can be a useful tool for learning the
syntax rules of a language.  Starting with a working program
(like hello\_world.c), modify it in various ways and see what happens.
If you get an error message, try to remember what the message says
and what caused it, so if you see it again in the future you
will know what it means.





\section{Glossary}

\begin{description}

\item[problem-solving:]  The process of formulating a problem, finding
a solution, and expressing the solution.

\item[high-level language:]  A programming language like C that
is designed to be easy for humans to read and write.

\item[low-level language:]  A programming language that is designed
to be easy for a computer to execute.  Also called ``machine
language'' or ``assembly language.''

\item[formal language:]  Any of the languages people have designed
for specific purposes, like representing mathematical ideas or
computer programs.  All programming languages are formal languages.

\item[natural language:]  Any of the languages people speak that
have evolved naturally.

\item[portability:]  A property of a program that can run on more
than one kind of computer.

\item[interpret:]  To execute a program in a high-level language
by translating it one line at a time.

\item[compile:]  To translate a program in a high-level language
into a low-level language, all at once, in preparation for later
execution.

\item[source code:]  A program in a high-level language, before
being compiled.

\item[object code:]  The output of the compiler, after translating
the program.

\item[executable:]  Another name for object code that is ready
to be executed.

\item[statement:] A part of a program that specifies an action
that will be performed when the program runs.  A print statement
causes output to be displayed on the screen.

\item[comment:] A part of a program that contains information
about the program, but that has no effect when the program runs.

\item[algorithm:]  A general process for solving a category of
problems.

\item[bug:]  An error in a program.

\item[syntax:]  The structure of a program.

\item[semantics:]  The meaning of a program.

\item[parse:]  To examine a program and analyze the syntactic structure.

\item[syntax error:]  An error in a program that makes it impossible
to parse (and therefore impossible to compile).

%\item[exception:]  An error in a program that makes it fail at
%run-time.  Also called a run-time error.

\item[logical error:]  An error in a program that makes it do something
other than what the programmer intended.

\item[debugging:]  The process of finding and removing any of
the three kinds of errors.

\index{problem-solving}
\index{high-level language}
\index{low-level language}
\index{formal language}
\index{natural language}
\index{interpret}
\index{compile}
\index{syntax}
\index{semantics}
\index{parse}
%\index{exception}
\index{error}
\index{debugging}
\index{statement}
\index{comment}

\end{description}

\section{Exercises}

% LaTeX source for textbook ``How to think like a computer scientist''
% Copyright (C) 1999  Allen B. Downey
% Copyright (C) 2009  Thomas Scheffler

\begin{exercise}

Computer scientists have the annoying habit of using common
English words to mean something different from their common
English meaning.  For example, in English, a statement and
a comment are pretty much the same thing, but when we are
talking about a program, they are very different.

The glossary at the end of each chapter is intended to highlight
words and phrases that have special meanings in computer science.
When you see familiar words, don't assume that you know what 
they mean!

\begin{enumerate}

\item In computer jargon, what's the difference between a statement
and a comment?

\item What does it mean to say that a program is portable?

\item What is an executable?

\end{enumerate}

\end{exercise}

\begin{exercise}

Before you do anything else, find out how to compile and run a C
program in your environment.  Some environments provide sample programs
similar to the example in Section~\ref{hello}.

\begin{enumerate}

\item Type in the ``Hello World'' program, then compile and run it.

\item Add a second print statement that prints a second message after
the ``Hello World.''.  Something witty like, ``How are you?''
Compile and run the program again.

\item Add a comment line to the program (anywhere) and recompile
it.  Run the program again.  The new comment should not affect the
execution of the program.

\end{enumerate}

This exercise may seem trivial, but it is the starting place for many
of the programs we will work with.  In order to debug with confidence,
you have to have confidence in your programming environment.  In some
environments, it is easy to lose track of which program is executing,
and you might find yourself trying to debug one program while you are
accidentally executing another.  Adding (and changing) print statements
is a simple way to establish the connection between the program you
are looking at and the output when the program runs.

\end{exercise}


\begin{exercise}

It is a good idea to commit as many errors as you can think of,
so that you see what error messages the compiler produces.
Sometimes the compiler will tell you exactly what is wrong, and all
you have to do is fix it.  Sometimes, though, the compiler will produce
wildly misleading messages.  You will develop a sense for when you can
trust the compiler and when you have to figure things out yourself.

\begin {enumerate}

\item Remove the closing curly-bracket (\}).

\item Remove the opening curly-bracket (\{).

\item Remove the {\tt int} before {\tt main}.

\item Instead of {\tt main}, write {\tt mian}.

\item Remove the closing {\tt */} from a comment.

\item Replace {\tt printf} with {\tt pintf}.

% Example of a logical error:
%
%\item Replace {\tt printf} with {\tt print}.  This one is
%tricky because it is a logical error, not a syntax error.
%The statement {\tt System.out.print} is legal, but it may or may
%not do what you expect.

\item Delete one of the parentheses:  {\tt (} or  {\tt )}.  Add an extra one.

\item Delete the semicolon after the {\tt return} statement.

\end {enumerate}
\end{exercise}


% LaTeX source for textbook ``How to think like a computer scientist''
% Copyright (C) 1999  Allen B. Downey
% Copyright (C) 2009  Thomas Scheffler

\setcounter{chapter}{1}
\chapter{Variables and types}

\section{More output}
\index{output}
\index{statement!output}

As I mentioned in the last chapter, you can put as many statements as
you want in {\tt main()}.  For example, to output more than one line:

\begin{verbatim}

  #include <stdio.h>
  #include <stdlib.h>

  /* main: generate some simple output */

  int main (void)
  {
        printf ("Hello World.\n");		    /* output one line */
        printf ("How are you?\n");		    /* output another line */       
        return (EXIT_SUCCESS);
  }

\end{verbatim}
%
As you can see, it is legal to put comments at the
end of a line, as well as on a line by themselves.

\index{String}
\index{type!String}

The phrases that appear in quotation marks are called {\bf strings},
because they are made up of a sequence (string) of letters.  Actually,
strings can contain any combination of letters, numbers, punctuation
marks, and other special characters.

\index{newline}

Often it is useful to display the output from multiple output
statements all on one line.  You can do this by leaving out
the {\tt $\backslash$n} from the first {\tt printf}:

\begin{verbatim}
  int main (void)
  {
        printf ("Goodbye, ");
        printf ("cruel world!\n");	     
        return (EXIT_SUCCESS);
  }
\end{verbatim}
%
In this case the output appears on a single line as
{\tt Goodbye, cruel world!}.  Notice that there is a space
between the word ``Goodbye,'' and the second quotation mark.
This space appears in the output, so it affects the behavior
of the program.

Spaces that appear outside of quotation marks generally do
not affect the behavior of the program.  For example, I
could have written:

\begin{verbatim}
  int main(void)
  {
  printf("Goodbye, ");
  printf("cruel world!\n");	     
  return(EXIT_SUCCESS);
  }
\end{verbatim}
%
This program would compile and run just as well as the original.
The breaks at the ends of lines (newlines) do not affect
the program's behavior either, so I could have written:

\begin{verbatim}
  int main(void){printf("Goodbye, ");printf("cruel world!\n");
  return(EXIT_SUCCESS);}
\end{verbatim}
%
That would work, too, although you have probably noticed that
the program is getting harder and harder to read.  Newlines and
spaces are useful for organizing your program visually, making
it easier to read the program and locate syntax errors.

\section{Values}
\index{value}
\index{type}

Computer programs operate on values stored in 
computer memory.
A value ---like a letter or
a number--- is one of the fundamental things that a program manipulates.  
The only values we have
manipulated so far are the strings we have been outputting, like
{\tt "Hello, world."}.  You (and the compiler) can identify
these string values because they are enclosed in quotation marks.
%constant values

There are different kinds of values, including integers and characters.
It is important for the program to know exactly what kind of value
is manipulated because not all manipulations will make sense on all
values.
We therefore distinguish between different {\bf types} of values.  
%Example 'a' + 'a'

An integer is a whole number like 1 or 17.  You can output
integer values in a similar way as you output strings:

\begin{verbatim}
   printf("%i\n", 17);
\end{verbatim}
%
When we look at the \texttt{printf()} statement more closely, we
notice that the value we are outputting no longer appears
inside the quotes, but behind them separated by comma.
The string is still there, but now contains a {\tt \%i} instead of
any text. 
The {\tt \%i} a placeholder that tells the \texttt{printf()} command
to print an integer value. Several such placeholders exist
for different data types and formatting options of the output.
We will see the next one just now.

A character value is a letter or digit or punctuation mark
enclosed in single quotes, like {\tt 'a'} or {\tt '5'}.
You can output character values in a similar way:

\begin{verbatim}
   printf("%c\n", '}');
\end{verbatim}
%
This example outputs a single closing curly-bracket on a line
by itself. It uses the {\tt \%c} placeholder to signify the output of a character
value.

It is easy to confuse different types of values, like {\tt "5"}, {\tt
'5'} and {\tt 5}, but if you pay attention to the punctuation, it
should be clear that the first is a string, the second is a character
and the third is an integer.  The reason this distinction is important
should become clear soon.

\section {Variables}
\index{variable}
\index{value}

One of the most powerful features of a programming language is the
ability to manipulate values through the use of {\bf variables}.  So far
the values that we have used in our statements where fixed to what 
was written in the statement. Now we will use a variable as a named 
location that stores a value.  

Just as there are different types of values (integer, character,
etc.), there are different types of variables.  When you create a new
variable, you have to declare what type it is.  For example, the
character type in C is called {\tt char}.  The following statement
creates a new variable named {\tt fred} that has type {\tt char}.

\begin{verbatim}
    char fred;
\end{verbatim}
%
This kind of statement is called a {\bf declaration}.

The type of a variable determines what kind of values it can
store.  A {\tt char} variable can contain characters, and it should
come as no surprise that {\tt int} variables can store integers.

Contrary to other programming languages, C does not have a 
dedicated variable type for the storage of string values. We will see in
a later chapter how string values are stored in C. 
%but we
%are going to skip that for now (see Chapter~\ref{strings}).

\index{declaration}
\index{statement!declaration}

To create an integer variable, the syntax is 

\begin{verbatim}
    int bob;
\end{verbatim}
%
where {\tt bob} is the arbitrary name you choose to identify the
variable.  In general, you will want to make up variable names
that indicate what you plan to do with the variable.  For
example, if you saw these variable declarations:

\begin{verbatim}
    char first_letter;
    char last_letter;
    int hour, minute;
\end{verbatim}
%
you could probably make a good guess at what values
would be stored in them.  This example
also demonstrates the syntax for declaring multiple variables
with the same type: {\tt hour} and {\tt minute}
are both integers ({\tt int} type).

ATTENTION: The older C89 standard allows variable declarations
only at the beginning of a block of code. It is therefore necessary
to put variable declarations before any other statements,
even if the variable itself is only needed much later in your program.

\section{Assignment}
\index{assignment}
\index{statement!assignment}

Now that we have created some variables, we would like to
store values in them.  We do that with an {\bf assignment
statement}.

\begin{verbatim}
    first_letter = 'a';   /* give first_letter the value 'a' */
    hour = 11;            /* assign the value 11 to hour */
    minute = 59;          /* set minute to 59 */
\end{verbatim}
%
This example shows three assignments, and the comments show
three different ways people sometimes talk about assignment
statements.  The vocabulary can be confusing here, but the
idea is straightforward:

\begin{itemize}

\item When you declare a variable, you create a named storage location.

\item When you make an assignment to a variable, you give it a value.

\end{itemize}

A common way to represent variables on paper is to draw a box
with the name of the variable on the outside and the value
of the variable on the inside.  This kind of figure is called
a {\bf state diagram} because is shows what state each 
variable is in (you can think of it as the variable's ``state of
mind'').
This diagram shows the effect of the three assignment statements:

%\vspace{0.1in}
%\centerline{\epsfig{figure=figs/assign.eps}}
%\vspace{0.1in}

\setlength{\unitlength}{1mm}
\begin{picture}(20,17)
\put(7,12){\large \texttt{first\_letter}}
\put(46,12){\large \texttt{hour}}
\put(74,12){\large \texttt{minute}}
\put(10,0){\framebox(20,10){{\large \textsf{a}}}}
\put(40,0){\framebox(20,10){{\large \textsf{11}}}}
\put(70,0){\framebox(20,10){{\large \textsf{59}}}}
\end{picture}

When we assign values to variables, we have to make sure that
the assigned value correspondents to the type of the variable.
In C  a variable has to have the same type as the
value you assign.  For example, you cannot store a string in
an {\tt int} variable.  The following statement generates a compiler
warning:

\begin{verbatim}
    int hour;
    hour = "Hello.";       /* WRONG !! */
\end{verbatim}
%
This rule is sometimes a source of confusion, because there are many
ways that you can convert values from one type to another, and C
sometimes converts things automatically.  But for now you should
remember that as a general rule variables and values have the same
type, and we'll talk about special cases later.

Another source of confusion is that some strings {\em look}
like integers, but they are not.  For example,
the string {\tt "123"}, which is made up of the
characters {\tt 1}, {\tt 2} and {\tt 3}, is not
the same thing as the {\em number} {\tt 123}.
This assignment is illegal:

\begin{verbatim}
    minute = "59";         /* WRONG!! */
\end{verbatim}
%
\section{Outputting variables}
\label{output variables}

You can output the value of a variable using the same commands
we used to output simple values.

\begin{verbatim}

    int hour, minute;
    char colon;

    hour = 11;
    minute = 59;
    colon = ':';

    printf ("The current time is ");
    printf ("%i", hour);
    printf ("%c", colon);
    printf ("%i", minute);
    printf ("\n"); 
  
\end{verbatim}
%
This program creates two integer variables named {\tt hour} and {\tt
minute}, and a character variable named {\tt colon}.  It assigns
appropriate values to each of the variables and then uses a series
of output statements to generate the following:

\begin{verbatim}
    The current time is 11:59
\end{verbatim}

When we talk about ``outputting a variable,'' we mean outputting the
{\em value} of the variable.  The name of a variable only has significance for
the programmer. The compiled program no longer contains a human readable
reference to the variable name in your program. 

The \texttt{ printf()} command is capable of outputting several variables
in a single statement. To do this, we need to put placeholders
in the so called \emph{format string}, that indicate the position where the variable value will
be put. The variables will be inserted in the order of their appearance in 
the statement. It is important to observe the right order and type for the variables.

By using a single output statement, we can make the previous program more
concise:

\begin{verbatim}
    
    int hour, minute;
    char colon;

    hour = 11;
    minute = 59;
    colon = ':';

    printf ("The current time is %i%c%i\n", hour, colon, minute);
    
\end{verbatim}
%
On one line, this program outputs a string, two integers and a character.  Very impressive!

\section{Keywords}
\index{keyword}

A few sections ago, I said that you can make up any name you
want for your variables, but that's not quite true.  There
are certain words that are reserved in C because they are
used by the compiler to parse the structure of your program,
and if you use them as variable names, it will get confused.
These words, called {\bf keywords}, include {\tt int},
{\tt char}, {\tt void} and many more.

\vskip 1em

\setlength{\fboxsep}{6pt} 
\begin{center}
\begin{boxedminipage}[c]{.9\linewidth}
\begin{center}
\begin{multicols}{5}[\underline{Reserved keywords in the C language}]
\begin{verbatim}
auto 
break 
case 
char 
const 
continue 
default 
do 
double 
else 
enum 
extern 
float 
for 
goto 
if 
inline 
int 
long 
register 
restrict 
return 
short 
signed 
sizeof 
static 
struct 
switch 
typedef 
union 
unsigned 
void 
volatile 
while 
_Bool 
_Complex 
_Imaginary 
\end{verbatim}
\end{multicols}
\end{center}
\end{boxedminipage}
\end{center}

\vskip 1em
The complete list of keywords is included in the C Standard, which
is the official language definition adopted by the the International
Organization for Standardization (ISO) on September 1, 1998.  

%You can download a copy electronically from
%
%\begin{verbatim}
%    http://www.ansi.org/
%\end{verbatim}
%
Rather than memorize the list, I would suggest that you
take advantage of a feature provided in many development
environments: code highlighting.  As you type, different
parts of your program should appear in different colors.  For
example, keywords might be blue, strings red, and other code
black.  If you type a variable name and it turns blue, watch
out!  You might get some strange behavior from the compiler.

\section{Operators}
\label{operators}
\index{operator}


{\bf Operators} are special symbols that are used to represent
simple computations like addition and multiplication.  Most
of the operators in C do exactly what you would expect them
to do, because they are common mathematical symbols.  For
example, the operator for adding two integers is {\tt +}.

The following are all legal C expressions whose meaning is
more or less obvious:

\begin{verbatim}
    1+1        hour-1       hour*60+minute     minute/60
\end{verbatim}
%
{\bf Expressions} can contain both variables
names and values.  In each case the name of the variable is
replaced with its value before the computation is performed.

\index{expression}

Addition, subtraction and multiplication all do what you
expect, but you might be surprised by division.  For example,
the following program:

\begin{verbatim}

  int hour, minute;
  hour = 11;
  minute = 59;
  printf ("Number of minutes since midnight: %i\n", hour*60 + minute);
  printf ("Fraction of the hour that has passed: %i\n", minute/60);

\end{verbatim}
%
would generate the following output:

\begin{verbatim}
    Number of minutes since midnight: 719
    Fraction of the hour that has passed: 0
\end{verbatim}
%
The first line is what we expected, but the second line is
odd.  The value of the variable {\tt minute} is 59, and
59 divided by 60 is 0.98333, not 0.  The reason for the
discrepancy is that C is performing {\bf integer division}.

\index{type!int}
\index{integer division}
\index{arithmetic!integer}
\index{division!integer}
\index{operand}

When both of the {\bf operands} are integers (operands are the things
operators operate on), the result must also be an integer,
and by definition integer division always rounds {\em down},
even in cases like this where the next integer is so close.

A possible alternative in this case is to calculate a percentage
rather than a fraction:

\begin{verbatim}
    printf ("Percentage of the hour that has passed: ");
    printf ("%i\n", minute*100/60);
\end{verbatim}
%
The result is:

\begin{verbatim}
    Percentage of the hour that has passed: 98
\end{verbatim}
%
Again the result is rounded down, but at least now the answer
is approximately correct.  In order to get an even more accurate
answer, we could use a different type of variable, called
floating-point, that is capable of storing fractional values.
We'll get to that in the next chapter.

\section{Order of operations}
\index{precedence}
\index{order of operations}

When more than one operator appears in an expression the order
of evaluation depends on the rules of {\bf precedence}.  A
complete explanation of precedence can get complicated, but
just to get you started:

\begin{itemize}

\item Multiplication and division happen before
addition and subtraction.  So {\tt 2*3-1} yields 5, not 4, and {\tt
2/3-1} yields -1, not 1.
%(remember that in integer division {\tt 2/3} is 0).

\item If the operators have the same precedence they are evaluated
from left to right.  So in the expression {\tt minute*100/60},
the multiplication happens first, yielding {\tt 5900/60}, which
in turn yields {\tt 98}.  If the operations had gone from right
to left, the result would be {\tt 59*1} which is {\tt 59}, which
is wrong.

\item Any time you want to override the rules of precedence (or
you are not sure what they are) you can use parentheses.  Expressions
in parentheses are evaluated first, so {\tt 2*(3-1)} is 4.
You can also use parentheses to make an expression easier to
read, as in {\tt (minute*100)/60}, even though it doesn't
change the result.

\end{itemize}

\section{Operators for characters}
\index{character operator}
\index{operator!character}

Interestingly, the same mathematical operations that work on
integers also work on characters.  For example,

\begin{verbatim}
    char letter;
    letter = 'a' + 1;
    printf ("%c\n", letter);
\end{verbatim}
%
outputs the letter {\tt b}.  Although it is syntactically legal
to multiply characters, it is almost never useful to do it.

Earlier I said that you can only assign integer values to
integer variables and character values to character variables,
but that is not completely true.  In some cases, C converts
automatically between types.  For example, the following is
legal.

\begin{verbatim}
    int number;
    number = 'a';
    printf ("%i\n", number);
\end{verbatim}
%
The result is 97, which is the number that is used internally
by C to represent the letter {\tt 'a'}.  However, it is
generally a good idea to treat characters as characters, and
integers as integers, and only convert from one to the other
if there is a good reason.

Automatic type conversion is an example of a common problem in designing a
programming language, which is that there is a conflict between {\bf
formalism}, which is the requirement that formal languages should have
simple rules with few exceptions, and {\bf convenience}, which is the
requirement that programming languages be easy to use in practice.

More often than not, convenience wins, which is usually good for
expert programmers, who are spared from rigorous but unwieldy
formalism, but bad for beginning programmers, who are often baffled
by the complexity of the rules and the number of exceptions.  In this
book I have tried to simplify things by emphasizing the rules and
omitting many of the exceptions.


\section{Composition}
\index{composition}
\index{expression}

So far we have looked at the elements of a programming
language---variables, expressions, and statements---in
isolation, without talking about how to combine them.

One of the most useful features of programming languages
is their ability to take small building blocks and
{\bf compose} them.  For example, we know how to multiply
integers and we know how to output values; it turns out we can
do both at the same time:

\begin{verbatim}
    printf ("%i\n", 17 * 3);
\end{verbatim}
%
Actually, I shouldn't say ``at the same time,'' since in reality
the multiplication has to happen before the output, but
the point is that any expression, involving numbers, characters,
and variables, can be used inside an output statement.  We've
already seen one example:

\begin{verbatim}
    printf ("%i\n", hour * 60 + minute);
\end{verbatim}
%
You can also put arbitrary expressions on the right-hand
side of an assignment statement:

\begin{verbatim}
    int percentage;
    percentage = (minute * 100) / 60;
\end{verbatim}
%
This ability may not seem so impressive now, but we will see
other examples where composition makes it possible
to express complex computations neatly and concisely.

WARNING: There are limits on where you can use certain
expressions; most notably, the left-hand side of an assignment
statement has to be a {\em variable} name, not an expression.
That's because the left side indicates the storage location
where the result will go.  Expressions
do not represent storage locations, only values.  So the
following is illegal:  {\tt minute + 1 = hour;}.

\section{Glossary}

\begin{description}

\item[variable:] A named storage location for values.  All
variables have a type, which determines which values it can
store.

\item[value:] A letter, or number, or other thing that can be
stored in a variable.  

\item[type:] The meaning of values.  The types
we have seen so far are integers ({\tt int} in C) and characters ({\tt
char} in C).

\item[keyword:]  A reserved word that is used by the compiler
to parse programs.  Examples we have seen include {\tt int},
{\tt void} and {\tt char}.

\item[statement:] A line of code that represents a command or
action.  So far, the statements we have seen are declarations,
assignments, and output statements.

\item[declaration:] A statement that creates a new variable and
determines its type.

\item[assignment:] A statement that assigns a value to a variable.

\item[expression:] A combination of variables, operators and
values that represents a single result value.  Expressions also
have types, as determined by their operators and operands.

\item[operator:] A special symbol that represents a simple
computation like addition or multiplication.

\item[operand:] One of the values on which an operator operates. 

\item[precedence:] The order in which operations are evaluated.

\item[composition:] The ability to combine simple
expressions and statements into compound statements and expressions
in order to represent complex computations concisely.

\index{variable}
\index{value}
\index{type}
\index{keyword}
\index{statement}
\index{assignment}
\index{expression}
\index{operator}
\index{operand}
\index{composition}

\end{description}


\section{Exercises}
\setcounter{exercisenum}{0}

% LaTeX source for textbook ``How to think like a computer scientist''
% Copyright (C) 1999  Allen B. Downey
% Copyright (C) 2009  Thomas Scheffler

\begin{exercise}
\label{ex.date}

\begin{enumerate}

\item Create a new program named {\tt MyDate.c}.  Copy or
type in something like the "Hello, World" program and make
sure you can compile and run it.

\item Following the example in Section~\ref{output variables}, write a program
that creates variables named {\tt day}, {\tt month}
and {\tt year}
What type is each variable?

Assign values to those variables that represent today's date.

\item Print the value of each variable on a line by itself.  This is
an intermediate step that is useful for checking that everything is
working so far.

\item Modify the program so that it prints the date in standard
American form: {\tt {\tt mm/dd/yyyy}}.

\item Modify the program again so that the total output is:

\begin{verbatim}
American format:
3/18/2009
European format:
18.3.2009
\end{verbatim}

\end{enumerate}

The point of this exercise is to use the output function {\tt printf} to display
values with different types, and to
practice developing programs gradually by adding a few statements
at a time.



\end{exercise}


\begin{exercise}

\begin{enumerate}

\item Create a new program called {\tt MyTime.c}.  From now
on, I won't remind you to start with a small, working program,
but you should.

\item Following the example in Section~\ref{operators}, create variables
named {\tt hour}, {\tt minute} and {\tt second}, and assign
them values that are roughly the current time.  Use a 24-hour
clock, so that at 2pm the value of {\tt hour} is 14.

\item Make the program calculate and print the number of
seconds since midnight.

\item Make the program calculate and print the number of
seconds remaining in the day.

\item Make the program calculate and print the percentage of
the day that has passed.

\item Change the values of {\tt hour}, {\tt minute} and {\tt second}
to reflect the current time (I assume that some time has elapsed), and
check to make sure that the program works correctly with different
values.

\end{enumerate}

The point of this exercise is to use some of the arithmetic
operations, and to start thinking about compound entities like the
time of day that are represented with multiple values.  Also,
you might run into problems computing percentages with {\tt ints},
which is the motivation for floating point numbers in the next
chapter.

HINT: you may want to use additional variables to hold values
temporarily during the computation.  Variables like this, that
are used in a computation but never printed, are sometimes called
intermediate or temporary variables.

\end{exercise}







\include{Chapter3}
\include{Chapter4}
\include{Chapter5}
\include{Chapter6_Iteration}
% LaTeX source for textbook ``How to think like a computer scientist''
% Copyright (C) 1999  Allen B. Downey
% Copyright (C) 2009  Thomas Scheffler


\chapter{Arrays}
\label{arrays}
\index{arrays}
\index{type!array}

A {\bf array} is a set of values where each value is identified and referenced by a
number (called an index).  The nice thing
about arrays is that they can be made up of any type of element,
including basic types like {\tt int}s and {\tt double}s, 
but all the values in an array have to have the same type.
%and user-defined types like {\tt Point} and {\tt Time}.


When you declare an array, you have to determine the number of
elements in the array. Otherwise the declaration looks similar to other variable types:

\begin{verbatim}
    int c[4];
    double values[10];
\end{verbatim}


%

Syntactically, array variables look like other C variables except that they are followed 
by {\tt [NUMBER\_OF\_ELEMENTS]}, the number of elements in the array enclosed in square brackets. 
The first line in our example, {\tt int c[4];} is of the type "array of integers" and creates a array of four integers named {\tt c}.
The second line, {\tt double values[10];} has the type "array of doubles" and
  creates an array of 10 {\tt double}s. 

%The number
%of elements in {\tt values} depends on {\tt size}. You can use any
%integer expression to determine the size of an array.
%!!! Not in C
%this would be dynamic arrays, that can not be initialised at definition time!!!

%

C allows you to to initialize the element values of an array immediately
after you have declared it.  The values  for the individual elements must be 
enclosed in curly brakets {\tt \{\}} and separated by comma, as in the following example:

\begin{verbatim}
    int c[4] = {0, 0, 0, 0};
\end{verbatim}

This statement creates an array of four elements and initializes
all of them to zero.
This syntax is only legal at initialisation time. Later in your program you can only
assign values for the array element by element.

%
The following figure shows how arrays are represented in state
diagrams:

%\myfig{figure=figs/array.eps}

\unitlength0.1cm

\begin{picture}(40,10)(-30,-5)
%\put(-4,1.5){{\Large \texttt{c}}}
%\put(0,1.5){\framebox(2,2)}
%\thicklines
%\put(2,2.5){\vector(1,0){8}}
%\thinlines
\put(5,1.5){{\Large \texttt{c}}}
\put(10,0){\framebox(7,5){\textbf{\textsf{0}}}}
\put(17,0){\framebox(7,5){\textbf{\textsf{0}}}}
\put(24,0){\framebox(7,5){\textbf{\textsf{0}}}}
\put(31,0){\framebox(7,5){\textbf{\textsf{0}}}}

\put(10.5,-4){{\scriptsize \texttt{c[0]}}}
\put(17.5,-4){{\scriptsize \texttt{c[1]}}}
\put(24.5,-4){{\scriptsize \texttt{c[2]}}}
\put(31.5,-4){{\scriptsize \texttt{c[3]}}}

\end{picture}

The large numbers inside the boxes are the values of the {\bf elements} in
the array.  The small numbers outside the boxes are the
indices used to identify each box.  When you allocate a new
array, without initializing, the arrays elements typically
contain arbitrary values and you must initialise them to
a meaningful value before using them.


%%
\section{Increment and decrement operators}
\index{operator!increment}
\index{operator!decrement}
\index{increment}
\index{decrement}

Incrementing and decrementing are such common operations that C
provides special operators for them.  The {\tt ++} operator adds one
to the current value of an {\tt int}, {\tt char} or {\tt double}, and
{\tt -}{\tt -} subtracts one.  
%Neither operator works on {\tt apstring}s,
%and neither {\em should} be used on {\tt bool}s.

Technically, it is legal to increment a variable and use it
in an expression at the same time.  For example, you might see
something like:

\begin{verbatim}
    printf ("%i\n ", i++);
\end{verbatim}
%
Looking at this, it is not clear whether the increment will
take effect before or after the value is displayed.  Because
expressions like this tend to be confusing, I would discourage
you from using them.  In fact, to discourage you even more,
I'm not going to tell you what the result is.  If you really
want to know, you can try it.

Using the increment operators, we can rewrite the {\tt PrintMultTable()} from Section~\ref{More generalization}:

\begin{verbatim}
    void PrintMultTable(int high) 
    { 
        int i = 1; 
        while (i <= high) 
        { 
            PrintMultiples(i); 
            i++; 
        } 
    }
\end{verbatim}
%
It is a common error to write something like:

\begin{verbatim}
    index = index++;             /* WRONG!! */
\end{verbatim}
%
Unfortunately, this is syntactically legal, so the compiler
will not warn you.  The effect of this statement is to leave
the value of {\tt index} unchanged.  This is often a difficult
bug to track down.

Remember, you can write {\tt index = index + 1;}, or you
can write {\tt index++;}, but you shouldn't mix them.

%%
\section{Accessing elements}
\index{element}
\index{array!element}

The {\tt []} operator allows us to read and write the individual elements of an array.  
The indices start at zero, so {\tt c[0]}
refers to the first element of the array, and {\tt c[1]}
refers to the second element.  You can use the {\tt []} operator
anywhere in an expression:


\begin{verbatim}
    c[0] = 7;
    c[1] = c[0] * 2;
    c[2]++;
    c[3] -= 60;
\end{verbatim}
%
All of these are legal assignment statements.  Here is the
effect of this code fragment:


%\myfig{figure=figs/array2.eps}

\unitlength0.1cm

\begin{picture}(40,10)(-30,-5)
%\put(-11,1.5){\texttt{count}}
%\put(0,1.5){\framebox(2,2)}
%\thicklines
%\put(2,2.5){\vector(1,0){8}}
%\thinlines
\put(5,1.5){{\Large \texttt{c}}}

\put(10,0){\framebox(7,5){\textbf{\textsf{7}}}}
\put(17,0){\framebox(7,5){\textbf{\textsf{14}}}}
\put(24,0){\framebox(7,5){\textbf{\textsf{1}}}}
\put(31,0){\framebox(7,5){\textbf{\textsf{-60}}}}

\put(10.5,-4){{\scriptsize \texttt{c[0]}}}
\put(17.5,-4){{\scriptsize \texttt{c[1]}}}
\put(24.5,-4){{\scriptsize \texttt{c[2]}}}
\put(31.5,-4){{\scriptsize \texttt{c[3]}}}

\end{picture}

By now you should have noticed that the four elements of this array
are numbered from 0 to 3, which means that there is no element with
the index 4.  

Nevertheless, it is a common error to go
beyond the bounds of an array. In safer languages such as Java, this will cause an 
error and most likely the program quits. C does not check array boundaries, so
your program can go on accessing memory locations beyond the array itself, as if 
they where part of the array. This is most likely wrong and can cause very
severe bugs in your program. 
\begin{quote}

{\bf It is necessary that you, as a programmer,
make sure that your code correctly observes array boundaries!}

\end{quote}




\index{run-time error}
\index{index}
\index{expression}

You can use any expression as an index, as long as it has type {\tt
int}.  One of the most common ways to index an array is with a loop
variable.  For example:

\begin{verbatim}
    int i = 0;
    while (i < 4) 
    {
        printf ("%i\n", c[i]);
        i++;
    }
\end{verbatim}

%
This is a standard {\tt while} loop that counts from 0
up to 4, and when the loop variable {\tt i} is 4, the
condition fails and the loop terminates.  Thus, the body
of the loop is only executed when {\tt i} is 0, 1, 2 and 3.

\index{loop}
\index{loop variable}
\index{variable!loop}

Each time through the loop we use {\tt i} as an index into
the array, printing the {\tt i}th element.  This type of
array traversal is very common.  Arrays and loops go together
like fava beans and a nice Chianti.

\index{fava beans}
\index{Chianti}

%%
\section{Copying arrays}
\index{array!copying}

Arrays can be a very convenient solution for a number of problems, like
storing and processing large sets of data.

However, there is very little that C does automatically for you. For example
you can not set all the elements of an array at the same time and you can not
assign one array to the other, even if they are identical in type and number of elements.


\begin{verbatim}
    double a[3] = {1.0, 1.0, 1.0};
    double b[3];

    a = 0.0;     /* Wrong! */
    b = a;       /* Wrong! */
\end{verbatim}
%

In order to set all of the elements of an array to some value, you must do so element by element.
To copy the contents of one array to another, you must again do so, by copying each element from
one array to the other.

\begin{verbatim}
    int i = 0;
    while (i < 3) 
    {
        b[i] = a[i];
        i++;
    }
\end{verbatim}

%%
\section{{\tt for} loops}

The loops we have written so far have a number of elements
in common.  All of them start by initializing a variable;
they have a test, or condition, that depends on that variable;
and inside the loop they do something to that variable,
like increment it.

\index{loop!for}
\index{for}
\index{statement!for}

This type of loop is so common that there is an alternate
loop statement, called {\tt for}, that expresses it more
concisely.  The general syntax looks like this:

\begin{verbatim}
    for (INITIALIZER; CONDITION; INCREMENTOR) 
    {
        BODY
    }
\end{verbatim}
%
This statement is exactly equivalent to

\begin{verbatim}
    INITIALIZER;
    while (CONDITION) 
    {
        BODY
        INCREMENTOR
    }
\end{verbatim}
%
except that it is more concise and, since it puts all the
loop-related statements in one place, it is easier to read.
For example:

\begin{verbatim}
    int i;
    for (i = 0; i < 4; i++) 
    {
        printf("%i\n", c[i]);
    }
\end{verbatim}
%
is equivalent to 

\begin{verbatim}
    int i = 0;
    while (i < 4) 
    {
        printf("%i\n", c[i]);
        i++;
    }
\end{verbatim}

%%
\section{Array length}
\label{Array length}
\index{length!array}
\index{array!length}
\index{sizeof}
\index{operator!sizeof}

C does not provide us with a convenient way to determine the
actual length of an array. Knowing the size of an array would
be convenient when we are looping through all elements of
the array and need to stop with the last element.

In order to determine the array length we could use the {\tt sizeof()} 
operator, that calculates the size of data types in bytes.
Most data types in C use more than one byte to store their values,
therefore it becomes necessary to divide the byte-count for the array by 
the byte-count for a single element to establish the number of elements
in the array.
\begin{verbatim}
    sizeof(ARRAY)/sizeof(ARRAY_ELEMENT)
\end{verbatim}

It is a good idea to use this value as the upper bound of a loop,
rather than a constant.  That way, if the size of the array
changes, you won't have to go through the program changing all the
loops; they will work correctly for any size array.

\begin{verbatim}
    int i, length;
    length = sizeof (c) / sizeof (c[0]);

    for (i = 0; i < length; i++) 
    {
        printf("%i\n", c[i]);
    }
\end{verbatim}
%
The last time the body of the loop gets executed, the value of {\tt i}
is {\tt length - 1}, which is the index of the last element.  When
{\tt i} is equal to {\tt length}, the condition fails and the body
is not executed, which is a good thing, since it would access a
memory location that is not part of the array.

\section{Random numbers}
\label{Random numbers}
\label{random}
\label{pseudorandom}
\index{random number}
\index{deterministic}
\index{nondeterministic}

Most computer programs do the same thing every time they are executed,
so they are said to be {\bf deterministic}.  Usually, determinism is a
good thing, since we expect the same calculation to yield the same
result.  For some applications, though, we would like the
computer to be unpredictable.  Games are an obvious example.

Making a program truly {\bf nondeterministic} turns out to be not
so easy, but there are ways to make it at least seem
nondeterministic.  One of them is to generate {pseudorandom} numbers and
use them to determine the outcome of the program.
Pseudorandom numbers
are not truly random in the mathematical sense, but 
for our purposes, they will do.

\index{header file!stdlib.h}
\index{<stdlib.h>}

C provides a function called {\tt rand()} that generates
pseudorandom numbers.  It is declared in the
header file {\tt stdlib.h}, which contains a variety of ``standard
library'' functions, hence the name.

The return value from {\tt rand()} is an integer between 0 and {\tt
RAND\_MAX}, where {\tt RAND\_MAX} is a large number (about 2 billion
on my computer) also defined in the header file.  Each time you call
{\tt rand()} you get a different randomly-generated number.  To see a
sample, run this loop:

\begin{verbatim}
    for (i = 0; i < 4; i++) 
    {
        int x = rand();
        printf("%i\n", x);
    }
\end{verbatim}
%
On my machine I got the following output:

\begin{verbatim}
1804289383
846930886
1681692777
1714636915
\end{verbatim}
%
You will probably get something similar, but different, on yours.

Of course, we don't always want to work with gigantic integers.
More often we want to generate integers between 0 and some
upper bound.  A simple way to do that is with the modulus
operator.  For example:

\begin{verbatim}
    int x = rand ();
    int y = x % upperBound;
\end{verbatim}
%
Since {\tt y} is the remainder when {\tt x} is divided by
{\tt upperBound}, the only possible values for {\tt y}
are between 0 and {\tt upperBound - 1}, including both
end points.  Keep in mind, though, that {\tt y} will never
be equal to {\tt upperBound}.

It is also frequently useful to generate random floating-point values.
A common way to do that is by dividing by {\tt RAND\_MAX}.  For
example:

\begin{verbatim}
    int x = rand ();
    double y = (double) x / RAND_MAX;
\end{verbatim}
%
This code sets {\tt y} to a random value between 0.0 and 1.0,
including both end points.  As an exercise, you might want to
think about how to generate a random floating-point value in
a given range; for example, between 100.0 and 200.0.

%%
\pagebreak
\section{Statistics}
\index{statistics}
\index{distribution}
\index{mean}

The numbers generated by {\tt rand()} are supposed to be distributed
uniformly.  That means that each value in the range should be
equally likely.  If we count the number of times each value appears,
it should be roughly the same for all values, provided that we
generate a large number of values.

In the next few sections, we will write programs that generate
a sequence of random numbers and check whether this property
holds true.

%%
\section{Array of random numbers}
\label{Array of random numbers}

The first step is to generate a large number of random values
and store them in a array.  By ``large number,'' of course,
I mean 20.  It's always a good idea to start with a manageable
number, to help with debugging, and then increase it later.

The following function takes three arguments, an array of integers, 
the size of the array and an upper bound for the random values.  
It fills the array of {\tt int}s with random values between 0 and {\tt upperBound-1}.

\begin{verbatim}
    void RandomizeArray (int array[], int length, int upperBound) 
    {
        int i;
        for (i = 0; i < length; i++) 
        {
            array[i] = rand() % upperBound;
        }
    }
\end{verbatim}
%
The return type is {\tt void}, which means that
this function does not return any value to the calling function.
To test this function, it is convenient to have a function that
outputs the contents of a array.

\begin{verbatim}
    void PrintArray (int array[], int length) 
    {
        int i;
        for (i = 0; i < length; i++) 
        {
            printf ("%i ",  array[i]);
        }
    }
\end{verbatim}
%
The following code generates an array filled with random values and outputs it:

\begin{verbatim}
    int r_array[20];
    int upperBound = 10;
    int length = sizeof(r_array) / sizeof(r_array[0]);
  
    RandomizeArray (r_array, length, upperBound);
    PrintArray (r_array, length);
\end{verbatim}

%
On my machine the output is:

\begin{verbatim}
3 6 7 5 3 5 6 2 9 1 2 7 0 9 3 6 0 6 2 6 
\end{verbatim}
\nopagebreak%
which is pretty random-looking.  Your results may differ.

If these numbers are really random,
we expect each digit to appear the same number of times---twice
each.  In fact, the number 6 appears five times, and the numbers 4
and 8 never appear at all.

Do these results mean the values are not really uniform?  It's
hard to tell.  With so few values, the chances are slim
that we would get exactly what we expect.  But as the number
of values increases, the outcome should be more predictable.

To test this theory, we'll write some programs that count the
number of times each value appears, and then see what happens
when we increase the number of elements in our array.

%%
\section{Passing an array to a function}
\label{Passing an array to a function}
\index{call by reference}
\index{call by value}
\index{array parameters}
You probably have noticed that our {\tt RandomizeArray()} function 
looked a bit unusual. We pass an array to this function and expect 
to get a a randomized array back. Nevertheless, we have declared it to 
be a {\tt void} function, and miraculously the function appears to have 
altered the array.

This behaviour goes against everything what I have said about the
use of variables in functions so far.
C typically uses the so called {\bf call-by-value} evaluation of
expressions. If you pass a value to a function it gets copied from
the calling function to a variable in the called function. The same
is true if the function returns a value.
Changes to the internal variable in the called function do not affect the external 
values of the calling function.

When we pass an array to a function this behaviour changes to
something called {\bf call-by-reference} evaluation.
C does not copy the array to an internal array --  it rather generates a
reference to the original array and any operation in the called function 
directly affects the original array.
This is also the reason why we do not have to return anything from our 
function. The changes have already taken place. 

Call by reference also makes it necessary to supply the length of
the array to the called function, since invoking  the {\tt sizeof}
operator in the called function would determine the size of the reference
and not the original array.

%!!!Reference to later chapter needed!!!
We will further discuss call by reference and call by value in 
Section~\ref{Pointers and Addresses}, Section~\ref{Call by value} and
\ref{Call by reference}.

%%
\section{Counting}
\label{counting}
\index{traverse!counting}
\index{loop!counting}
\index{counter}

A good approach to problems like this is to think of simple functions
that are easy to write, and that might turn out to be useful.  Then
you can combine them into a solution.  This approach is sometimes
called {\bf bottom-up design}.  

Of course, it is not easy to
know ahead of time which functions are likely to be useful, but as you
gain experience you will have a better idea.
\index{bottom-up design}
\index{program development!bottom-up}
Also, it is not always obvious what sort of things are easy to write,
but a good approach is to look for subproblems that fit a pattern you
have seen before.

\index{pattern!counter}

%Back in Section~\ref{loopcount} we looked at a loop that traversed a
%string and counted the number of times a given letter appeared.  
In our current example we want to examine a potentially large set
of elements and count the number of times a certain value appears.
You
can think of this program as an example of a pattern called ``traverse
and count.''  The elements of this pattern are:

\begin{itemize}

\item A set or container that can be traversed, like a string
or a array.

\item A test that you can apply to each element in the container.

\item A counter that keeps track of how many elements pass
the test.

\end{itemize}

In this case, I have a function in mind called {\tt HowMany()} that
counts the number of elements in a array that are equal to a given value.
The parameters are the array, the length of the array and the integer value we are looking
for.  The return value is the number of times the value appears.

\begin{verbatim}
    int HowMany (int array[], int length, int value) 
    {
        int i; 
        int count = 0;
  
        for (i=0; i < length; i++) 
            {
                if (array[i] == value) count++;
            }
        return count;
    }
\end{verbatim}


\section{Checking the other values}

{\tt HowMany()} only counts the occurrences of a particular value, and
we are interested in seeing how many times each value appears.
We can solve that problem with a loop:

\begin{verbatim}
    int i;
    int r_array[20];
    int upperBound = 10;
    int length = sizeof(r_array) / sizeof(r_array[0]);
  
    RandomizeArray(r_array, length, upperBound);

    printf ("value\tHowMany\n");
    for (i = 0; i < upperBound; i++) 
    {
        printf("%i\t%i\n", i, HowMany(r_array, length, i));
    }
\end{verbatim}
%

%! ! ! Applies only to C++! ! !
%Notice that it is legal to declare a variable inside a {\tt for}
%statement.  This syntax is sometimes convenient, but you should
%be aware that a variable declared inside a loop only exists
%inside the loop.  If you try to refer to {\tt i} later, you
%will get a compiler error.

This code uses the loop variable as an argument to
{\tt HowMany()}, in order to check each value between 0 and 9,
in order.  The result is:

\begin{verbatim}
value   HowMany
0       2
1       1
2       3
3       3
4       0
5       2
6       5
7       2
8       0
9       2
\end{verbatim}
%
Again, it is hard to tell if the digits are really appearing
equally often.  If we increase the size of the array to 100,000 we
get the following:

\begin{verbatim}
value   HowMany
0       10130
1       10072
2       9990
3       9842
4       10174
5       9930
6       10059
7       9954
8       9891
9       9958
\end{verbatim}
%
In each case, the number of appearances is within about 1\% of
the expected value (10,000), so we conclude that the random
numbers are probably uniform.

\section {A histogram}
\index{histogram}

It is often useful to take the data from the previous tables
and store them for later access, rather than just print them.
What we need is a way to store 10 integers.  We could create
10 integer variables with names like {\tt howManyOnes},
{\tt howManyTwos}, etc.  But that would require a lot of
typing, and it would be a real pain later if we decided to
change the range of values.

A better solution is to use a array with length 10.  That
way we can create all ten storage locations at once and we
can access them using indices, rather than ten different names.
Here's how:

\begin{verbatim}
    int i;
    int upperBound = 10;
    int r_array[100000];
    int histogram[upperBound];
    int r_array_length = sizeof(r_array) / sizeof(r_array[0]);
  
    RandomizeArray(r_array, r_array_length, upperBound);

    for (i = 0; i < upperBound; i++) 
    {
        int count = HowMany(r_array, length, i);
        histogram[i] = count;
    }  
\end{verbatim}
%
I called the array {\bf histogram} because that's
a statistical term for a array of numbers that counts the
number of appearances of a range of values.

\index{histogram}

The tricky thing here is that I am using the loop variable
in two different ways.  First, it is an argument to {\tt HowMany()},
specifying which value I am interested in.  Second, it is
an index into the histogram, specifying which location I should
store the result in.

\section{A single-pass solution}

Although this code works, it is not as efficient as it could
be.  Every time it calls {\tt HowMany()}, it traverses the
entire array.  In this example we have to traverse the
array ten times!

It would be better to make a single pass through the array.
For each value in the array we could find the corresponding
counter and increment it.  In other words, we can use the
value from the array as an index into the histogram.  Here's
what that looks like:

\begin{verbatim}
    int upperBound = 10;
    int histogram[upperBound] = {0, 0, 0, 0, 0, 0, 0, 0, 0, 0};

    for (i = 0; i < r_array_length; i++) 
    {
        int index = r_array[i];
        histogram[index]++;
    }
\end{verbatim}
%
The second line initializes the elements of the histogram to
zeroes.  That way, when we use the increment
operator ({\tt ++}) inside the loop, we know we are starting from zero.
Forgetting to initialize counters is a common error.

As an exercise, encapsulate this code in a function called {\tt Histogram()} 
that takes an array and the range of values in the array
(in this case 0 through 10) as two parameters \texttt{min} and \texttt{max}. 
You should pass a second array to the function where a histogram of the
values in the array can be stored.

\section{Random seeds}
\label{Random seeds}
\index{seed}
\index{random}

If you have run the code in this chapter a few times, you might
have noticed that you are getting the same ``random'' values
every time.  That's not very random!

One of the properties of pseudorandom number generators is that
if they start from the same place they will generate
the same sequence of values.  The starting place is called
a {\bf seed}; by default, C uses
the same seed every time you run the program.

While you are debugging, it is often helpful to
see the same sequence over and over.  That way, when you make
a change to the program you can compare the output before and
after the change.

If you want to choose a different seed for the random number
generator, you can use the {\tt srand()} function.  It takes
a single argument, which is an integer between 0 and {\tt RAND\_MAX}.

For many applications, like games, you want to see a different
random sequence every time the program runs.  A common way to
do that is to use a library function like {\tt time()}
to generate something reasonably unpredictable
and unrepeatable, like the number of seconds since January
1970, and use that number as a seed.  The details
of how to do that depend on your development environment.

\section{Glossary}

\begin{description}

\item[array:]  A named collection of values, where all the
values have the same type, and each value is identified by
an index.

\item[element:]  One of the values in a array.  The {\tt []}
operator selects elements of a array.

\item[index:]  An integer variable or value used to indicate
an element of a array.

\item[increment:]  Increase the value of a variable by one.
The increment operator in C is {\tt ++}. 

\item[decrement:]  Decrease the value of a variable by one.
The decrement operator in C is {\tt -}{\tt -}.

\item[deterministic:]  A program that does the same thing every
time it is run.

\item[pseudorandom:]  A sequence of numbers that appear to be
random, but which are actually the product of a deterministic
computation.

\item[seed:]  A value used to initialize a random number sequence.
Using the same seed should yield the same sequence of values.

\item[bottom-up design:]  A method of program development that
starts by writing small, useful functions and then assembling
them into larger solutions.

\item[histogram:]  A array of integers where each integer
counts the number of values that fall into a certain range.

\index{array}
\index{element}
\index{index}
\index{deterministic}
\index{pseudorandom}
\index{seed}
\index{histogram}

\end{description}

%%
\section{Exercises}
\setcounter{exercisenum}{0}


% LaTeX source for textbook ``How to think like a computer scientist''
% Copyright (C) 1999  Allen B. Downey
% Copyright (C) 2009  Thomas Scheffler

%%%%%%%%%%%%%%%%%%%%%%%%%%%%%%%%%%%%%%
\begin{exercise}

A friend of yours shows you the following method and
explains that if {\tt number} is any two-digit number, the program
will output the number backwards.  He claims that if {\tt number} is
17, the method will output {\tt 71}.

Is he right?  If not, explain what the program actually does and
modify it so that it does the right thing.

\begin{verbatim}
   #include <stdio.h>
   #include<stdlib.h>
   
   int main (void)
   {
     int number = 71;
     int lastDigit = number%10;
     int firstDigit = number/10;
     printf("%i",lastDigit + firstDigit);
     return EXIT_SUCCESS;
  }

\end{verbatim}

\end{exercise}


%%%%%%%%%%%%%%%%%%%%%%%%%%%%%%%%%%%%%%
\begin{exercise}
Write a function that takes an array of integers, the length of the array and an integer named
{\tt target} as arguments. 

The function should search through the provided array and should return the first index where
{\tt target} appears in the array, if it does. If {\tt target} is not in the array the function should 
return an invalid index value to indicate an error condition  (e.g.  -1).
\end{exercise}

%%%%%%%%%%%%%%%%%%%%%%%%%%%%%%%%%%%%%%

\begin{exercise}

One not-very-efficient way to sort the elements of an array
is to find the largest element and swap it with the first
element, then find the second-largest element and swap it with
the second, and so on.

\begin{enumerate}

\item Write a function called {\tt IndexOfMaxInRange()} that 
takes an array of integers, finds the
largest element in the given range, and returns its {\em index}.

\item Write a function called {\tt SwapElement()} that takes an
array of integers and two indices, and that swaps the elements
at the given indices.

\item Write a function called {\tt SortArray()} that takes an array of
integers and that uses {\tt IndexOfMaxInRange()} and {\tt SwapElement()}
to sort the array from largest to smallest.

\end{enumerate}
\end{exercise}


%%%%%%%%%%%%%%%%%%%%%%%%%%%%%%%%%%%%%%




% LaTeX source for textbook ``How to think like a computer scientist''
% Copyright (C) 1999  Allen B. Downey
% Copyright (C) 2009  Thomas Scheffler


\selectlanguage{english}
\chapter{Strings and things}
\label{strings}

\section{Containers for strings}

We have seen four types of values---characters, integers,
floating-point numbers and strings---but only three types of
variables---{\tt char}, {\tt int} and {\tt double}.  So
far we have no way to store a string in a variable or perform
operations on strings.

This chapter is going to rectify this situation and I can now tell
you that strings in C are stored as an array of characters terminated by the 
character {\tt \textbackslash 0}.

By now this explanation should make sense to you and you probably understand
why we had to learn quite a bit about the working of the language 
before we could turn our attention towards string variables.

\index{<string.h>}
\index{header file!string.h}
In the previous chapter we have seen that operations on arrays have 
only minimal support from the C language itself and we had
to program extra functions by ourselves.
Fortunately things are a little bit easier when we manipulate
these special types of arrays - called strings.  There exist a number of
library functions in {\tt string.h} that make string handling a bit easier
than operations on pure arrays. 

Nevertheless string operations in 
C are still a lot more cumbersome than their equivalence in other
programing languages and can be a 
potential source of errors in your programs, if not handled 
carefully.

\section{String variables}

You can create a string variable as an array of characters in the following
way:

\begin{verbatim}
    char first[] = "Hello, ";
    char second[] = "world.";
\end{verbatim}
%
The first line creates an {\tt string} and assigns it the string value {\tt "Hello."}
In the second line we declare a second string variable. Remember,
the combined declaration and assignment is called initialization.

Initialisation time is the only time you can assign a value to a string directly (just as with
arrays in general). The initialisation parameters are passed 
in the form of a string constant enclosed in quotation marks ({\tt"}\ldots {\tt"}).

Notice the difference in syntax for the initialisation of arrays and strings. 
If you like you can also initialize the string in the normal array syntax, 
although this looks a little odd and is not very convenient to type.  

\begin{verbatim}
    char first[] = {'H','e','l','l','o',',',' ','\0'};
\end{verbatim}

There is no need to supply an array size when you are initialising the 
string variable at declaration time. The compiler compute the necessary
array size to store the supplied string.

Remember what we said about the nature of a string variable. It is an array of
characters \textbf{plus} a marker that shows where our string ends: the 
termination character  {\tt \textbackslash 0}.

Normally you do not have to supply this termination character.
The compiler understands our code and insertes it automatically.
However, in the example above, we treated our string exactly
like an array and in this case we have to insert the termination character ourselves.

When we are using a string variable to store different sting values 
during the lifetime of our program we have to declare a size big enough
for the largest sequence of characters that we are going to store.
We also have to make our string variable exactly one character longer than
the text we are going to store, because of the necessary termination character. 

We can output strings in the usual way using the {\tt printf()} function:

\begin{verbatim}
    printf("%s", first);
\end{verbatim}



%%
\section{Extracting characters from a string}

Strings are called ``strings'' because they are made up of a sequence,
or string, of characters.  The first operation we are going to
perform on a string is to extract one of the characters.  C
uses an index in square brackets ({\tt [} and {\tt ]}) for this operation:

\begin{verbatim}
    char fruit[] = "banana";
    char letter = fruit[1];
    printf ("%c\n", letter);
\end{verbatim}
%
The expression {\tt fruit[1]} indicates that I want character number 1
from the string named {\tt fruit}.  The result is stored in a {\tt
char} named {\tt letter}.  When I output the value of {\tt letter}, I
get a surprise:

\begin{verbatim}
   a
\end{verbatim}
%
{\tt a} is not the first letter of {\tt "banana"}.  Unless you are a
computer scientist.  For perverse reasons, computer scientists always
start counting from zero.  The 0th letter (``zeroeth'') of {\tt
"banana"} is {\tt b}.  The 1th letter (``oneth'') is {\tt a} and the
2th (``twoeth'') letter is {\tt n}.

If you want the zereoth letter of a string, you have to put
zero in the square brackets:

\begin{verbatim}
    char letter = fruit[0];
\end{verbatim}

\section{Length}
\index{string!length}
\index{length!string}
\index{<string.h>}
\index{header file!string.h}

To find the length of a string (the number of characters this string contains), we can
use the {\tt strlen()} function.  The function is called using the string variable
as an argument:

\begin{verbatim}
    #include <string.h>
    int main(void)
    { 
       int length;
       char fruit[] = "banana"; 

       length = strlen(fruit);
       return EXIT_SUCCESS;
    }   
\end{verbatim}
%
The return value of {\tt strlen()} in this case is 6. We assign this value to the integer
 {\tt length}  for further use.  

In order to compile this code, you need to include the
header file for the {\tt string.h} library. This library provides a number of
useful functions for operations on strings. 
%The type of the {\tt strlen()} is {\tt size_t}, an unsigned value large enough to
%enumerate any object that the system can handle (such as a string).
%for our example we can safely assume that the size of the string object 
%in our example will never
%exceed the range of the integer type. 
You should familiarize yourself with these functions because they can 
help you to solve your programming problems faster.


%Notice
%that it is legal to have a variable with the same name as a function.

To find the last letter of a string, you might be tempted to
try something like

\begin{verbatim}
    int length = strlen(fruit);
    char last = fruit[length];       /* WRONG!! */
\end{verbatim}
%
That won't work.  The reason is that  {\tt fruit} is still an array and there is no letter
at the array index  {\tt fruit[6]} in {\tt "banana"}.  Since we started counting at 0, the 6
letters are numbered from 0 to 5.  To get the last character,
you have to subtract 1 from {\tt length}.

\begin{verbatim}
    int length = strlen(fruit);
    char last = fruit[length-1];
\end{verbatim}

\section{Traversal}
\index{traverse}

A common thing to do with a string is
start at the beginning, select each character in turn, do
something to it, and continue until the end.  This pattern
of processing is called a {\bf traversal}.  A natural
way to encode a traversal is with a {\tt while} statement:

\begin{verbatim}
    int index = 0;
    while (index < strlen(fruit)) 
    {
        char letter = fruit[index];
        printf("%c\n" , letter);
        index = index + 1;
    }
\end{verbatim}
%
This loop traverses the string and outputs each letter on
a line by itself.  Notice that the condition is
{\tt index < strlen(fruit)}, which means that when
{\tt index} is equal to the length of the string, the
condition is false and the body of the loop is not executed.
The last character we access is the one with the
index {\tt strlen(fruit)-1}.

\index{loop variable}
\index{variable!loop}
\index{index}

The name of the loop variable is {\tt index}.  An {\bf
index} is a variable or value used to specify one member of an ordered
set, in this case the set of characters in the string.  The index
indicates (hence the name) which one you want.  The set has to be
ordered so that each letter has an index and each index
refers to a single character.

As an exercise, write a function that takes a {\tt string}
as an argument and that outputs the letters backwards, all on
one line.

%\section{A run-time error}
%\index{error!run-time}
%\index{run-time error}

%Way back in Section~\ref{run-time} I talked about run-time errors,
%which are errors that don't appear until a program has started
%running.

%So far, you probably haven't seen many run-time errors, because we
%haven't been doing many things that can cause one.  Well, now we are.
%If you use the {\tt []} operator and you provide an index that is
%negative or greater than {\tt length-1}, you will get a run-time
%error and a message something like this:

%\begin{verbatim}
%index out of range: 6, string: banana
%\end{verbatim}
%%
%Try it in your development environment and see how it looks.


%%
\section{Finding a  {\tt character} in a {\tt string}}
\label{Finding a  character in a string}

If we are looking for a letter in a {\tt string}, we have to search 
through the string and detect the position where this
letter occurs in the string.
Here is an implementation of this function:

\begin{verbatim}
    int LocateCharacter(char *s, char c)
    {
        int i = 0;
        while (i < strlen(s)) 
        {
            if (s[i] == c) return i;
            i = i + 1;
        }
        return -1;
    }
\end{verbatim}

We have to pass the {\tt string}
as the first argument, the other argument is the character
we are looking for. Our function returns the index of the first
occurrence of the letter, or {\tt -1} if the letter is not contained
in the string.


%%
\section{Pointers and Addresses}
\label{Pointers and Addresses}
\index{pointer}
\index{address}

When we look at the definition of the {\tt LocateCharacter()} function
you may notice the following construct {\tt char *s} which looks unfamiliar.

Remember, when we discussed how we had to pass
an array to a function, back in Section~\ref{Passing an array to a function},
we said that instead of copying the array, we only pass a reference to the function. 
Back then, we did not say exactly what this reference was.



C is one of the very few high-level programming languages that
let you directly manipulate objects in the computer memory.
In order to do this direct manipulation, we need to know the location 
of the object in memory: it's address. 
Adresses can be stored in variables of a special type.
These variables that point to other objects in memory 
(such as variables, arrays and strings) 
are therefore called {\bf pointer} variables. 

A pointer references the memory location of an object
and can be defined like this:

\begin{verbatim}
    int *i_p;
\end{verbatim}

This declaration looks similar to our earlier declarations, with one difference: the asterisk
in front of the name. 
We have given this pointer the type {\tt int}. The type specification has nothing to do
with the pointer itself, but rather defines which object this pointer is
supposed to reference (in this case an {\tt integer}).
This allows the compiler to do some type checking on, what would
otherwise be, an anonymous reference.
  
A pointer all by itself is rather meaningless, we also need an object that
this pointer is referencing:

\begin{verbatim}
    int number = 5; 
    int *i_p;
\end{verbatim}
 
This code-fragment defines an {\tt int} variable and a pointer. We can use 
 the "address-of" operator~{\tt \&} to assign  the memory 
 location or {\bf address} of our variable to the pointer.
 
 \begin{verbatim}
    i_p = &number;
\end{verbatim}

%{\tt}
Pointer {\tt i\_p} now references integer variable {\tt number}.
We can verify this using the "content-of" operator~{\tt *}.

 \begin{verbatim}
    printf("%i\n", *i_p);
\end{verbatim}

This prints {\tt 5}, which happens to be the content of the 
memory location at our pointer reference. 

With pointers we can directly manipulate memory locations:

 \begin{verbatim}
    *i_p = *i_p + 2;
    printf("%i\n", number);
\end{verbatim}

Our variable {\tt number} now has the value {\tt 7} and we begin to 
understand how our {\tt LocateCharacter()} function can directly access 
the values of string variables through the use of a {\tt char} pointer.

Pointers are widely used in many C programs and we have only
touched the surface of the topic. They can be immensely useful 
and efficient, however they can also be a potential source of
problems when not used appropriately. For this reason not
many programming languages support direct memory manipulation.
 

%If we are looking for a letter in an {\tt string}, we may
%not want to start at the beginning of the string.  One way
%to generalize the {\tt find} function is to write a version
%that takes an additional parameter---the index where we should
%start looking.  Here is an implementation of this function.

%\begin{verbatim}
%    int Find (char *s, char c, int i)
%    {
%        while (i < strlen(s)) 
%        {
%            if (s[i] == c) return i;
%            i = i + 1;
%        }
%        return -1;
%    }
%\end{verbatim}
%
%We have to pass the {\tt string}
%as the first argument.  The other arguments are the character
%we are looking for and the index where we should start.



%%
%\section{Looping and counting}
%\label{loopcount}
%\index{traverse!counting}
%\index{loop!counting}

%The following program counts the
%number of times the letter {\tt 'a'} appears in a string:

%\begin{verbatim}
%    char fruit[] = "banana";
%    int length = strlen(fruit);
%    int count = 0;

%    int index = 0;
%    while (index < length) 
%    {
%        if (fruit[index] == 'a') 
%        {
%            count ++;
%        }
%        index++;
%    }
%    printf ("%i\n", count);
%\end{verbatim}
%%
%This program demonstrates a common idiom, called a {\bf counter}.  The
%variable {\tt count} is initialized to zero and then incremented each
%time we find an {\tt 'a'}.  (To {\bf increment} is to increase by one;
%it is the opposite of {\bf decrement}.)  When we exit the loop, {\tt count}
%contains the result: the total number of a's.

%\index{counter}

%As an exercise, encapsulate this code in a function named
%{\tt CountLetters()}, and generalize it so that it accepts the
%string and the letter as arguments.
%% m�ssen wir die L�nge vorher ermitteln und �bergeben?

%\index{encapsulation}
%\index{generalization}

%As a second exercise, rewrite this function so that instead
%of traversing the string, it uses the version of
%{\tt find} we wrote in the previous section.


%\section{The {\tt strchr} function}
%\index{find}

%


% The {\tt strchr} function is like the opposite of the
%{\tt []} operator.  Instead of taking an index and extracting the
%character at that index, {\tt strchr} takes a character and finds the
%index where that character appears.

%\begin{verbatim}
%  char fruit[] = "banana";
%  int index = strchar(fruit,'a'));
%\end{verbatim}
%%
%This example finds the index of the letter {\tt 'a'} in the string.
%In this case, the letter appears three times, so it is not obvious
%what {\tt find} should do.  According to the documentation, it returns
%the index of the {\em first} appearance, so the result is 1.  If the
%given letter does not appear in the string, {\tt find} returns -1.

%In addition, there is a 
%version of {\tt find} that takes another {\tt string} as
%an argument and that finds the index where the substring
%appears in the string.  For example,

%\begin{verbatim}
%  apstring fruit = "banana";
%  int index = fruit.find("nan");
%\end{verbatim}
%%
%This example returns the value 2.




%%
%\pagebreak[4]

\section{String concatenation}

In Section~\ref{Finding a  character in a string} we have seen how we
could implement a search function that finds a {\tt character} in a {\tt string}.

One useful operation on strings is string {\bf concatenation}.  
To concatenate means to
join the two operands end to end.  For example:  {\tt shoe}
and {\tt maker} becomes {\tt shoemaker}.

Fortunately, we do not have to program all the necessary functions in C ourselves.
The {\tt string.h} library already provides several functions that we can
invoke on strings. 

We can use the library function {\tt strncat()} to concatenate
strings in C. 

\begin{verbatim}
    char fruit[20] = "banana";
    char bakedGood[] = " nut bread";
    strncat(fruit, bakedGood, 10);
    printf ("%s\n", fruit);
\end{verbatim}
%
The output of this program is {\tt banana nut bread}.

When we are using library functions it is important to completely understand
all the necessary arguments and to have a complete understanding
of the working of the function. 

The {\tt strncat()} does not take the two strings, joins them together and
produces a new combined string. It rather copies the content from the second
argument into the first. 

We therefore have to make sure that our first string is long enough to 
also hold the second string. We do this by defining the maximum capacity for 
string {\tt fruit} to be 19 characters + 1 termination character ({\tt char fruit[20]}). 
The third argument of {\tt strncat()}  specifies 
the number of characters
that will be copied from the second into the first string.



%It is also possible to concatenate a character onto the
%beginning or end of an {\tt string}.  In the following example, we
%will use concatenation and character arithmetic to output
%an abecedarian series.

%``Abecedarian'' refers to a series or list in which the elements
%appear in alphabetical order.  For example, in Robert McCloskey's book
%{\em Make Way for Ducklings}, the names of the ducklings are Jack,
%Kack, Lack, Mack, Nack, Ouack, Pack and Quack.  Here is a loop that
%outputs these names in order:

%\begin{verbatim}
%    char name[5];
%    char suffix[] = "ack";

%    char letter = 'J';
%    name[0] = letter;
%    name[1] = '\0';
%    
%    while (letter <= 'Q') 
%    {
%	/* Wrong, does not work, string gets longer and longer...*/        
%        printf("%s\n", strncat (name, suffix, 3));
%        letter++;
%        name[0] = letter;
%    }
%\end{verbatim}
%%
%The output of this program is:

%\begin{verbatim}
%Jack
%Kack
%Lack
%Mack
%Nack
%Oack
%Pack
%Qack
%\end{verbatim}
%%
%Of course, that's not quite right because I've misspelled ``Ouack''
%and ``Quack.''  As an exercise, modify the program to correct
%this error.

%Again, be careful to use string concatenation only with {\tt apstring}s
%and not with native C strings.  Unfortunately, an expression like
%{\tt letter + "ack"} is syntactically legal in C++, although it
%produces a very strange result, at least in my development environment.

%%
%\section{{\tt string}s are mutable}
%\index{immutable}
%\index{string}

%You can change the letters in an {\tt string} one at a time
%using the {\tt []} operator on the left side of an assignment.
%For example,

%\begin{verbatim}
%    char greeting[] = "Hello, world!";
%    greeting[0] = 'J';
%    printf ("%s", greeting);
%\end{verbatim}
%
%produces the output {\tt Jello, world!}.

\section{Assigning new values to {\tt string} variables}
\index{assigning!string}
\index{string}

So far we have seen how to initialise a string variable 
at declaration time. As with arrays in general, it is not 
legal to assign values directly to strings, because it is
not  possible to assign a value to an entire array.

\begin{verbatim}
    fruit = "orange";  /* Wrong: Cannot assign directly! */
\end{verbatim}

In order to assign a new value to an existing string variable we
have to use the {\tt strncpy()} function.
For example,

\begin{verbatim}
    char greeting[15];
    strncpy (greeting, "Hello, world!", 13);
\end{verbatim}

copies 13 characters from the of the second argument 
string to the first argument string.

This works, but not quite as expected. The {\tt strncpy()} function
copies exactly 13 characters from the second argument string
into the first argument string. And what happens to our 
string termination character  {\tt \textbackslash 0}?

%\pagebreak[4]

It is \textbf{not} copied automatically. We need to change
our copy statement to copy also the invisible 14th character at
the end of the string:

\begin{verbatim}
    strncpy (greeting, "Hello, world!", 14);
\end{verbatim}

However, if we only copy parts of the second string into the first
we need to explicitly set the n+1th character in the {\tt greeting[15]}
string to {\tt \textbackslash 0} afterwards.

\begin{verbatim}
    strncpy (greeting, "Hello, world!", 5); /*only Hello is copied*/
    greeting[5] = '\0';
\end{verbatim}

\vskip 1.5em

{\bf Attention!} In the last two sections we have used
the {\tt strncpy()} and the {\tt strncat()} function that require you
to explicitly supply the number of characters that will get copied
or attached to the first argument string. 

The {\tt string.h} library also defines the {\tt strcpy()} and 
the {\tt strcat()} functions that have no explicit bound on the number
of characters that are copied. 

The usage of these functions is
strongly discouraged! Their use has lead to a vast number
of security problems with C programs. Remember, C does not
check array boundaries and will continue copying characters
into computer memory even past the length of the variable.


%%
\section{{\tt string}s are not comparable}
\label{incomparable}
\index{comparison!string}
\index{string}

All the comparison operators that work on {\tt int}s and
{\tt double}s do work on {\tt strings}.  For example,
if you write the following code to determine if two strings are equal:

\begin{verbatim}
    if (word == "banana")  /* Wrong! */ 
\end{verbatim}

This test will always fail.

%
You have to use the {\tt strcmp()} function to compare two strings
with each other. The function returns {\tt 0} if the two strings are 
identical, a negative value if the first string is 'alphabetically less' than
the second (would be listed first in a dictionary) or a positive value
if the second string is 'greater'.

Please notice, this return value is not the standard true/false result, where the
return value {\tt 0}  is interpreted as 'false'.



The  {\tt strcmp()} function is useful for putting words
in alphabetical order.

\begin{verbatim}
    if (strcmp(word, "banana") < 0) 
    {
        printf( "Your word, %s, comes before banana.\n", word);
    } 
    else if (strcmp(word, "banana") > 0) 
    {
        printf( "Your word, %s, comes after banana.\n", word);
    } 
    else 
    {
        printf ("Yes, we have no bananas!\n");
    }
\end{verbatim}
%
You should be aware, though, that the {\tt strcmp()} function does
not handle upper and lower case letters the same way that people
do.  All the upper case letters come before all the lower case
letters.  As a result,

\begin{verbatim}
Your word, Zebra, comes before banana.
\end{verbatim}
%
A common way to address this problem is to convert strings to a
standard format, like all lower-case, before performing the
comparison.  The next sections explains how. 

%%
\section{Character classification}

\index{<ctype.h>}
\index{header file!ctype.h}
It is often useful to examine a character and test whether
it is upper or lower case, or whether it is a character or
a digit.  C provides a library of functions that perform
this kind of character classification.  In order to use these
functions, you have to include the header file {\tt ctype.h}.

\begin{verbatim}
    char letter = 'a';
    if (isalpha(letter)) 
    {
        printf("The character %c is a letter.", letter);
    }
\end{verbatim}
%
The return value from {\tt isalpha()} is an integer that is
0 if the argument is not a letter, and some non-zero value
if it is.

It is legal to use this kind of integer in a conditional, as shown
in the example.  The value {\tt 0} is treated as {\tt false}, and
all non-zero values are treated as {\tt true}.

%Technically, this sort of thing should not be allowed---integers are
%not the same thing as boolean values.  Nevertheless, the C habit of
%converting automatically between types can be useful.

Other character classification functions include {\tt isdigit()}, which
identifies the digits 0 through 9, and {\tt isspace()}, which identifies
all kinds of ``white'' space, including spaces, tabs, newlines, and a
few others.  There are also {\tt isupper()} and {\tt islower()}, which
distinguish upper and lower case letters.

Finally, there are two functions that convert letters from one
case to the other, called {\tt toupper()} and {\tt tolower()}.  Both take
a single character as an argument and return a (possibly
converted) character.

\begin{verbatim}
    char letter = 'a';
    letter = toupper (letter);
    printf("%c\n", letter);
\end{verbatim}
%
The output of this code is {\tt A}.

As an exercise, use the character classification and conversion
library to write functions named {\tt StringToUpper()} and
{\tt StringToLower()} that take a single string as
a parameter, and that modify the string by converting all the
letters to upper or lower case.  The return type should be
{\tt void}.

%%%
%\section{Other {\tt string} functions}

%This chapter does not cover all the {\tt apstring} functions.
%Two additional ones, {\tt c\_str} and {\tt substr}, are covered
%in Section~\ref{finput} and Section~\ref{parsing}.

\section{Getting user input}
\label{input}
\index{input!keyboard}

The programs we have written so far are pretty predictable;
they do the same thing every time they run.  Most of the time,
though, we want programs that take input from the user and
respond accordingly.

There are many ways to get input, including keyboard
input, mouse movements and button clicks, as well as more exotic
mechanisms like voice control and retinal scanning.  In this
text we will consider only keyboard input.

\index{scanf()}
\index{printf()}

In the header file {\tt stdio.h},
C defines a function named {\tt scanf()} that handles input in
much the same way that {\tt printf()} handles output.  We can use the following code to get an
integer value from the user:

\begin{verbatim}
    int x;
    scanf("%i", &x);
\end{verbatim}
%
The {\tt scanf()} function causes the program to stop executing and
wait for the user to type something.  If the user types a valid
integer, the program converts it into an integer value and
stores it in {\tt x}.

If the user types something other than an integer,
C doesn't report an error, or anything sensible like that.
Instead, the {\tt scanf()} function returns and leaves the value in {\tt x} unchanged.

Fortunately, there is a way to check and see if an input
statement succeeds.  The {\tt scanf()} function returns the number
of items that have been successfully read.
This number will be {\tt 1} when the last input
statement succeeded.  If not, we know that some previous operation
failed, and also that the next operation will fail.

Getting input from the user might look like this:

\begin{verbatim}
    int main (void)
    {
        int success, x;

        /* prompt the user for input */
        printf ("Enter an integer: \n");

        /* get input */
        success = scanf("%i", &x);

        /* check and see if the input statement succeeded */
        if (success == 1) 
        {
            /* print the value we got from the user */
            printf ("Your input: %i\n", x);
            return EXIT_SUCCESS;
        }
        printf("That was not an integer.\n");
        return EXIT_FAILURE;
    }
\end{verbatim}
%
There is another potential pitfall connected with the {\tt scanf()} function.
Your program code might want to insist that the user types a valid integer, because
this value is needed later on. In this case you might want to 
repeat the input statement in order to get a valid user input:
 
 \begin{verbatim}
    if (success != 1) 
    {
          while (success != 1)                                      
          { 
               printf("That was not a number. Please try again:\n");
               success = scanf("%i", &x);
          }  
     }
\end{verbatim}

\index{input!flushing the buffer}
\index{input buffer!flushing the buffer}
Unfortunately this code leads into an endless loop. You probably ask yourself, why?
The input from the keyboard is delivered to your program by the operating system, in 
something called an input buffer. A successful read operation automatically empties this buffer.
However, if the {\tt scanf()} function fails, like in our example, the buffer does not get emptied
and the next {\tt scanf()} operation re-reads the old value - you see the problem?

We need to empty the input buffer, before we can attempt to read the next input from the
user. Since there is no standard way to do this, we will introduce our own code that 
reads and empties the buffer using the {\tt getchar()} function. It run through a  {\tt while}-loop 
until there are no more characters left in the buffer (notice the construction of this loop, where all
the operations are executed in the test condition):
\begin{verbatim}
      char ch;    /* helper variable stores discarded chars*/
      while (success != 1)                                      
      { 
          printf("That isn't a number. Please try again:\n");
           /* now we empty the input buffer*/
           while ((ch = getchar()) != '\n' && ch != EOF);
           success = scanf("%i", &x);
      }    

\end{verbatim}
 

The {\tt scanf()} function can also be used to input a {\tt string}:

\begin{verbatim}
    char name[80];

    printf ("What is your name?");
    scanf ("%s", name);
    printf ("%s", name);
\end{verbatim}
%
Again, we have to make sure our string variable is large enough
to contain the complete user input. Notice the difference in the
argument of the {\tt scanf()} function when we are reading 
an {\tt integer} or a {\tt string}. The function requires a
pointer to the variable where the input value will be stored.
If we are reading an {\tt integer} we need to use the address operator {\tt \&}
with the variable name. In the case of a {\tt string} we simply provide the 
variable name.

Also notice, that the {\tt scanf()} function only takes the first word of
the input, and leaves the rest for the next input statement.
So, if you run this program and type your full name, it
will only output your first name.



\section{Glossary}

\begin{description}

\item[index:]  A variable or value used to select one of the
members of an ordered set, like a character from a string.

\item[traverse:]  To iterate through all the elements of a set
performing a similar operation on each.

\item[counter:]  A variable used to count something, usually
initialized to zero and then incremented.

\item[concatenate:] To join two operands end-to-end.

\item[pointer:] A reference to an object in computer memory.

\item[address:] The exact storage location of objects in memory.

\index{index}
\index{traverse}
\index{counter}
\index{increment}
\index{decrement}
\index{concatenate}
\index{pointer}
\index{address}


\end{description}

\section{Exercises}
\setcounter{exercisenum}{0}


% LaTeX source for textbook ``How to think like a computer scientist''
% Copyright (C) 1999  Allen B. Downey
% Copyright (C) 2009  Thomas Scheffler

%%%%%%%%%%%%%%%%%%%%%%%%%%%%%%%%%%%%%%
\begin{exercise}
A word is said to be ``abecedarian'' if the letters in the
word appear in alphabetical order.  For example, the following
are all 6-letter English abecedarian words.

\begin {quote}
abdest, acknow, acorsy, adempt, adipsy, agnosy, befist, behint,
beknow, bijoux, biopsy, cestuy, chintz, deflux, dehors, dehort,
deinos, diluvy, dimpsy
\end{quote}

\begin{enumerate}

\item Describe an algorithm for checking whether a given word (String)
is abecedarian, assuming that the word contains only lower-case
letters.  Your algorithm can be iterative or recursive.

\item Implement your algorithm in a function called {\tt IsAbecedarian()}.

\end{enumerate}
\end{exercise}

%%%%%%%%%%%%%%%%%%%%%%%%%%%%%%%%%%%%%%


\begin{exercise}
Write a function called {\tt LetterHist()} that takes a String as a
parameter and that returns a histogram of the letters in the String.
The zeroeth element of the histogram should contain the number of a's
in the String (upper and lower case); the 25th element should contain
the number of z's.
Your solution should only traverse the String once.
\end{exercise}


%%%%%%%%%%%%%%%%%%%%%%%%%%%%%%%%%%%%%%


\begin{exercise}
A word is said to be a ``doubloon'' if every letter that appears in the
word appears exactly twice.  For example, the following are all the
doubloons I found in my dictionary.

\begin {quote}
Abba, Anna, appall, appearer, appeases, arraigning, beriberi,
bilabial, boob, Caucasus, coco, Dada, deed, Emmett, Hannah,
horseshoer, intestines, Isis, mama, Mimi, murmur, noon, Otto, papa,
peep, reappear, redder, sees, Shanghaiings, Toto
\end{quote}

Write a function called {\tt IsDoubloon()} that returns {\tt TRUE}
if the given word is a doubloon and {\tt FALSE} otherwise.
\end{exercise}




%%%%%%%%%%%%%%%%%%%%%%%%%%%%%%%%%%

\begin{exercise}

The Captain Crunch decoder ring works by taking each letter in a
string and adding 13 to it.  For example, 'a' becomes 'n' and 'b'
becomes 'o'.  The letters ``wrap around'' at the end, so 'z' becomes
'm'.

\begin{enumerate}
\item  Write a function that takes a String and that returns a new String
containing the encoded version.  You should assume that the String
contains upper and lower case letters, and spaces, but no other
punctuation.  Lower case letters should be transformed into other lower
case letters; upper into upper.  You should not encode the spaces.

\item Generalize the Captain Crunch method so that instead of adding
13 to the letters, it adds any given amount.  Now you should be able
to encode things by adding 13 and decode them by adding -13.  Try it.

\end{enumerate}
\end{exercise}


%%%%%%%%%%%%%%%%%%%%%%%%%%%%%%%%%%

\begin{exercise}
In Scrabble each player has a set of tiles with letters on them, and
the object of the game is to use those letters to spell words.  The
scoring system is complicated, but as a rough guide longer words are
often worth more than shorter words.

Imagine you are given your set of tiles as a String, like {\tt
"qijibo"} and you are given another String to test, like {\tt "jib"}.
Write a function called {\tt TestWord()} that takes these two Strings and
returns true if the set of tiles can be used to spell the word.  You
might have more than one tile with the same letter, but you can only
use each tile once.
\end{exercise}



%%%%%%%%%%%%%%%%%%%%%%%%%%%%%%%%%%



\begin{exercise}
In real Scrabble, there are some blank tiles that can be used
as wild cards; that is, a blank tile can be used to represent
any letter.

Think of an algorithm for {\tt TestWord()} that deals with wild
cards.  Don't get bogged down in details of implementation like
how to represent wild cards.  Just describe the algorithm, using
English, pseudocode, or C.
\end{exercise}







% LaTeX source for textbook ``How to think like a computer scientist''
% Copyright (C) 1999  Allen B. Downey
% Copyright (C) 2009  Thomas Scheffler


\chapter{Structures}
\label{structs}
\index{struct}

\section{Compound values}

Most of the data types we have been working with represent a single
value---an integer, a floating-point number, a character value. 
Strings are different in the sense that they are made up of smaller
pieces, the characters.  Thus, strings are an example of a
{\bf compound} type. 

Depending on what we are doing, we may want to treat a compound type
as a single thing (or object), or we may want to access its parts (or
member variables).  This ambiguity is useful.

It is also useful to be able to create your own compound values.  
C provides a mechanism for doing that: {\bf structures}.  

\section{{\tt Point} objects}
\index{Point}
\index{struct!Point}

As a simple example of a compound structure, consider the concept of a
mathematical point.  At one level, a point is two numbers
(coordinates) that we treat collectively as a single object.  In
mathematical notation, points are often written in parentheses, with a
comma separating the coordinates.  For example, $(0, 0)$ indicates the
origin, and $(x, y)$ indicates the point $x$ units to the right and
$y$ units up from the origin.

A natural way to represent a point in C is with two {\tt double}s.
The question, then, is how to group these two values into
a compound object, or structure.  The answer is a {\tt struct}
definition:

\begin{verbatim}
    typedef struct 
    {
        double x;
        double y;
    } Point_t;  
\end{verbatim}
%
{\tt struct} definitions appear outside of any function definition,
usually at the beginning of the program (after the {\tt include}
statements).

This definition indicates that there are two elements in this
structure, named {\tt x} and {\tt y}.  These elements are called
the {\bf members} or {\bf fields} of a structure.
%, for reasons I will explain a little later.

It is a common error to leave off the semi-colon at the end of a
structure definition.  It might seem odd to put a semi-colon after
curly-brackets, but you'll get used to it.

Once you have defined the new structure, you can create variables
with that type:

\begin{verbatim}
    Point_t blank;
    blank.x = 3.0;
    blank.y = 4.0;   
\end{verbatim}
%
The first line is a conventional variable declaration: {\tt blank} has
type {\tt Point\_t}.  The next two lines initialize the fields of the
structure.  The ``dot notation'' used here is called the {\bf field selection
operator} and allows to access the structure fields.

\index{declaration}
\index{statement!declaration}
\index{reference}
\index{state diagram}
\index{state}

The result of these assignments is shown in the following
state diagram:

\vspace{0.1in}
\centerline{\epsfig{figure=figs/point.pdf, width=3.0cm}}
\vspace{0.1in}

As usual, the name of the variable {\tt blank} appears outside the box
and its value appears inside the box.  In this case, that value is
a compound object with two named member variables.

\section{Accessing member variables}
\index{struct!member variable}

You can read the values of an member variable using the same syntax we
used to write them:

\begin{verbatim}
    double x = blank.x;
\end{verbatim}
%
The expression {\tt blank.x} means ``go to the object named {\tt
blank} and get the value of {\tt x}.''  In this case we assign that
value to a local variable named {\tt x}.  Notice that there is no
conflict between the local variable named {\tt x} and the member
variable named {\tt x}.  The purpose of dot notation is to identify
{\em which} variable you are referring to unambiguously.

You can use dot notation as part of any C expression, so the
following are legal.

\begin{verbatim}
    printf ("%0.1f, %0.1f\n", blank.x, blank.y);
    double distance = blank.x * blank.x + blank.y * blank.y;
\end{verbatim}
%
The first line outputs {\tt 3, 4}; the second line calculates
the value 25.

\section{Operations on structures}
\index{struct!operations}
\index{typecasting}

Most of the operators we have been using on other types, like
mathematical operators ( {\tt +}, {\tt \%}, etc.) and comparison
operators ({\tt ==}, {\tt >}, etc.), do not work on structures.
%Actually, it is possible to define the meaning of these operators
%for the new type, but we won't do that in this book.

On the other hand, the assignment operator {\em does} work for
structures.  It can be used in two ways: to initialize the member
variables of a structure or to copy the member variables from one
structure to another.  An initialization looks like this:

\begin{verbatim}
    Point_t blank = { 3.0, 4.0 };
\end{verbatim}
%
The values in squiggly braces get assigned to the member variables of
the structure one by one, in order.  So in this case, {\tt x}
gets the first value and {\tt y} gets the second.

Unfortunately, this syntax can be used only in an initialization,
not in an assignment statement.  So the following is illegal:

\begin{verbatim}
    Point_t blank;
    blank = { 3.0, 4.0 };       /* WRONG !! */
\end{verbatim}
%
You might wonder why this perfectly reasonable statement should
be illegal; I'm not sure, but I think the problem is that the compiler
doesn't know what type the right hand side should be.  You must specify
the type of the assignment by adding a typecast:

\begin{verbatim}
    Point_t blank;
    blank = (Point_t){ 3.0, 4.0 };
\end{verbatim}
%
That works.

It is legal to assign one structure to
another.  For example:

\begin{verbatim}
    Point_t p1 = { 3.0, 4.0 };
    Point_t p2 = p1;
    printf ("%f, %f\n", p2.x, p2.y);
\end{verbatim}
%
The output of this program is {\tt 3, 4}.

%%
\section{Structures as parameters}
\label{Structures as parameters}
\index{parameter}
\index{struct!as parameter}

You can pass structures as parameters in the usual way.  For
example,

\begin{verbatim}
    void PrintPoint (Point_t point) 
    {
        printf ("(%0.1f, %0.1f)\n", point.x, point.y);
    }
\end{verbatim}
%
{\tt PrintPoint()} takes a point as an argument and outputs it in
the standard format.  If you call {\tt PrintPoint(blank)},
it will output {\tt (3.0, 4.0)}.

As a second example, we can rewrite the {\tt ComputeDistance()} function from
Section~\ref{distance} so that it takes two {\tt Point}s as parameters
instead of four {\tt double}s.

\begin{verbatim}
    double ComputeDistance (Point_t p1, Point_t p2) 
    {
        double dx = p2.x - p1.x;
        double dy = p2.y - p1.y;
        return sqrt (dx*dx + dy*dy);
    }
\end{verbatim}

\section{Call by value}
\label{Call by value}
\index{parameter passing}
\index{call by value}

When you pass a structure as an argument, remember that the
argument and the parameter are not the same variable.  Instead,
there are two variables (one in the caller and one in the
callee) that have the same value, at least initially.  For
example, when we call {\tt PrintPoint()}, the stack diagram
looks like this:

\vspace{0.1in}
\centerline{\epsfig{figure=figs/stack_point2.pdf, width=6cm}}
\vspace{0.1in}
%
If {\tt PrintPoint()} happened to change one of the member variables
of {\tt point}, it would have no effect on {\tt blank}.  Of course, there
is no reason for {\tt PrintPoint()} to modify its parameter, so this
isolation between the two functions is appropriate.

This kind of parameter-passing is called ``pass by value''
because it is the value of the structure (or other type) that
gets passed to the function.

\section{Call by reference}
\label{Call by reference}
\index{parameter passing}
\index{call by reference}
\index{reference}

An alternative parameter-passing mechanism that is available
in C is called ``pass by reference.''  
By now we already know that C uses pointers as references.
This mechanism makes
it possible to pass a structure to a procedure and modify it directly.

For example, you can reflect a point around the 45-degree line by
swapping the two coordinates.  The most obvious (but incorrect) way to
write a {\tt ReflectPoint()} function is something like this:

\begin{verbatim}
    void ReflectPoint (Point_t point)      /* Does not work! */
    {
        double temp = point.x;
        point.x = point.y;
        point.y = temp;
    }
\end{verbatim}
%
This won't work, because the changes we make in {\tt ReflectPoint()}
will have no effect on the caller.

Instead, we have to specify that we want to pass the parameter by
reference.  
Our function now has a struct pointer argument {\tt Point\_t~*ptr}.


\begin{verbatim}
    void ReflectPoint (Point_t *ptr)
    {
        double temp = ptr->x;
        ptr->x = ptr->y;
        ptr->y = temp;
    }
\end{verbatim}
When we are accessing the struct member variables through a pointer 
we can no longer use the "field-selection-operator" ({\tt .}). Instead we need to use
the "pointing-to" operator ({\tt ->}).

%
We pass a reference of our struct parameter by adding the "address-of"  operator ({\tt \&}) to the
structure variable when we call the function:

\begin{verbatim}
    PrintPoint (blank);
    ReflectPoint (&blank);
    PrintPoint (blank);
\end{verbatim}
%
The output of this program is as expected:

\begin{verbatim}
    (3.0, 4.0)
    (4.0, 3.0)
\end{verbatim}
%
Here's how we would draw a stack diagram for this program:

\vspace{0.1in}
\centerline{\epsfig{figure=figs/stack_point3.pdf, width=6.5cm}}
\vspace{0.1in}
%
The parameter {\tt ptr} is a reference to the structure named {\tt
blank}.  The usual representation for a reference is a dot with an
arrow that points to whatever the reference refers to.

The important thing to see in this diagram is that any changes that
{\tt ReflectPoint()} makes through {\tt ptr} will also affect {\tt blank}.

Passing structures by reference is more versatile than passing by
value, because the callee can modify the structure.  It is also
faster, because the system does not have to copy the whole
structure.  On the other hand, it is less safe, since it is harder to
keep track of what gets modified where.  Nevertheless, in C
programs, almost all structures are passed by reference almost all the
time.  In this book I will follow that convention.


\section{Rectangles}
\index{Rectangle}
\index{struct!Rectangle}

Now let's say that we want to create a structure to represent
a rectangle.  The question is, what information do I have to
provide in order to specify a rectangle?  To keep things simple
let's assume that the rectangle will be oriented vertically or
horizontally, never at an angle.

There are a few possibilities: I could specify the center of
the rectangle (two coordinates) and its size (width and height),
or I could specify one of the corners and the size, or I
could specify two opposing corners.

The most common choice in existing programs is to specify the
upper left corner of the rectangle and the size.  To do that
in C, we will define a structure that contains a {\tt Point\_t}
and two doubles.

\begin{verbatim}
    typedef struct 
    {
        Point_t corner;
        double width, height;
    } Rectangle_t;  
\end{verbatim}
%
Notice that one structure can contain another.  In fact, this
sort of thing is quite common.  Of course, this means that in
order to create a {\tt Rectangle\_t}, we have to create a {\tt Point\_t}
first:

\begin{verbatim}
    Point_t corner = { 0.0, 0.0 };
    Rectangle_t box = { corner, 100.0, 200.0 };
\end{verbatim}
%
This code creates a new {\tt Rectangle\_t} structure and initializes the
member variables.  The figure shows the effect of this assignment.

\vspace{0.1in}
\centerline{\epsfig{figure=figs/rectangle.pdf, width=6cm}}
\vspace{0.1in}
%
We can access the {\tt width} and {\tt height} in the usual way:

\begin{verbatim}
    box.width += 50.0;
    printf("%f\n", box.width);
\end{verbatim}
%
In order to access the member variables of {\tt corner}, we can use a
temporary variable:

\begin{verbatim}
    Point_t temp = box.corner;
    double x = temp.x;
\end{verbatim}
%
Alternatively, we can compose the two statements:

\index{composition}

\begin{verbatim}
    double x = box.corner.x;
\end{verbatim}
%
It makes the most sense to read this statement from right to
left: ``Extract {\tt x} from the {\tt corner} of the {\tt box},
and assign it to the local variable {\tt x}.''

While we are on the subject of composition, I should point
out that you can, in fact, create the {\tt Point} and the
{\tt Rectangle} at the same time:

\begin{verbatim}
    Rectangle_t box = { { 0.0, 0.0 }, 100.0, 200.0 };
\end{verbatim}
%
The innermost squiggly braces are the coordinates of the
corner point; together they make up the first of the three
values that go into the new {\tt Rectangle}.  This statement
is an example of {\bf nested structure}.

\index{nested structure}


\section{Structures as return types}
\index{struct!as return type}
\index{return}
\index{statement!return}

You can write functions that return structures.  For example,
{\tt FindCenter()} has a {\tt Rectangle\_t} parameter and
returns a {\tt Point\_t} that contains the coordinates of the
center of the rectangle:

\begin{verbatim}
    Point_t FindCenter (Rectangle_t box)
    {
        double x = box.corner.x + box.width/2;
        double y = box.corner.y + box.height/2;
        Point_t result = {x, y};
        return result;
    }
\end{verbatim}
%
To call this function, we have to pass a {\tt Rectangle\_t} as an argument
(notice that it is being passed by value), and assign the
return value to a {\tt Point\_t} variable:

\begin{verbatim}
    Rectangle_t box = { {0.0, 0.0}, 100, 200 };
    Point_t center = FindCenter (box);
    PrintPoint (center);
\end{verbatim}
%
The output of this program is {\tt (50, 100)}.

We could have passed the structure as a reference to the
function. In this case our function would look like this:

\begin{verbatim}
    Point_t FindCenter (Rectangle_t *box)
    {
        double x = box->corner.x + box->width/2;
        double y = box->corner.y + box->height/2;
        Point_t result = {x, y};
        return result;
    }
\end{verbatim}
Notice, how we had to change the access to the members of the 
structure, since {\tt box} is now a pointer. 
We would also have to change the function call for {\tt FindCenter()}:

\begin{verbatim}
    Point_t center = FindCenter (&box);
\end{verbatim}

\section {Passing other types by reference}
\index{parameter passing}
\index{call by reference}
\index{reference}

It's not just structures that can be passed by reference.
All the other types we've seen can, too.  For example, to swap
two integers, we could write something like:

\begin{verbatim}
    void Swap (int *x, int *y)
    {
        int temp = *x;
        *x = *y;
        *y = temp;
    }
\end{verbatim}
%
We would call this function in the usual way:

\begin{verbatim}
    int i = 7;
    int j = 9;
    printf (" i=%i, j=%i\n", i, j);
    Swap (&i, &j);
    printf (" i=%i, j=%i\n", i, j);
\end{verbatim}
%
The output of this program shows that the variable
values have been swapped.  Draw a stack
diagram for this program to convince yourself this is true.
If the parameters {\tt x} and {\tt y} were declared as
regular integer variables (without the {\tt \*}s), {\tt Swap()} would
not work.  It would modify {\tt x} and {\tt y} and have no
effect on {\tt i} and {\tt j}.

When people start passing things like integers by reference,
they often try to use an expression
as a reference argument.  For example:

\begin{verbatim}
    int i = 7;
    int j = 9;
    Swap (&i, &j+1);         /* WRONG!! */
\end{verbatim}
%
Presumably the programmer wanted to increase the value of {\tt j} by {\tt 1}
before it is passed to the function.
This does not work as expected, because the expression {\tt j+1} now
is interpreted a pointer value and in now pointing to a memory
location beyond the variable {\tt j}. 
It is a little tricky to figure out exactly
what kinds of expressions make sense to be passed by reference.  For now
a good rule of thumb is that reference arguments have to be
variables.


\section{Glossary}

\begin{description}

\item[structure:]  A collection of data grouped together and
treated as a single object.

\item[member variable:]  One of the named pieces of data that make up
a structure.

\item[reference:]  A value that indicates or refers to a variable
or structure.  In a state diagram, a reference appears as an arrow.

\item[pass by value:]  A method of parameter-passing in which the
value provided as an argument is copied into the corresponding
parameter, but the parameter and the argument occupy distinct
locations.

\item[pass by reference:]  A method of parameter-passing in which
the parameter is a reference to the argument variable.  Changes
to the parameter also affect the argument variable.

\index{structure}
\index{member variable}
\index{reference}
\index{pass by value}
\index{pass by reference}

\end{description}

\section{Exercises}
\setcounter{exercisenum}{0}

% LaTeX source for textbook ``How to think like a computer scientist''
% Copyright (C) 1999  Allen B. Downey
% Copyright (C) 2009  Thomas Scheffler

%%%%%%%%%%%%%%%%%%%%%%%%%%%%%%%%%%%%%%%%%%

\begin{exercise}

Section~\ref{Structures as parameters} defines the function {\tt PrintPoint()}. 
The argument of this function is passed along as a value (\emph{call-by-value}).

\begin{enumerate}
\item Change the definition of this function, so that it only passes a reference to 
the structure to the function for printing (\emph{call-by-reference}).

\item Test the new function with different values and document the results. 
\end{enumerate}


\end{exercise}


%%%%%%%%%%%%%%%%%%%%%%%%%%%%%%%%%%%%%%%%%%

\begin{exercise}

Write a program, that defines a struct \texttt{Person\_t}.
This struct should contain two members. The first member should be a string
of sufficient length, to contain the name of a person. The second member should
be an integer value containing the persons age. 

\begin{enumerate}
\item Define two struct variables \texttt{person1} and \texttt{person2}.\\ 
Initialize the two struct variables with suitable values. The first person
should be called Betsy. Betsy should be 42 years old.\\
The second person should be named after yourself and should also be as old as yourself.


\item Write a function \texttt{PrintPerson()}. The function should take a struct variable of type
\texttt{Person\_t} as an argument and print the name of the person and the corresponding age.
\item Write a function \texttt{HappyBirthday()}. The function should increase the age of the 
corresponding person by one year and print out a birthday greeting.\\
\textbf{Hint: }You need to passes a reference to the structure to the function (\emph{call-by-reference}). 
If your function uses \emph{call-by-value}, the age only changes in the local copy of the struct and
the changes you have made are lost, once the function terminates. Try it out, if you do not believe me!
\end{enumerate}


\end{exercise}




%%%%%%%%%%%%%%%%%%%%%%%%%%%%%%%%%%%%%%%%%%

\begin{exercise}

Most computer games can capture our interest only when their actions are
non-predictable, otherwise they become boring quickly.
Section~\ref{Random numbers} tells us how to generate random numbers in
in C. 


Write a simple game, where the computer chooses an arbitrary number in
the range between 1 and 20. You will then be asked to guess the
number chosen by the Computer.

To give you a hint the computer should answer your guess in the following
way: in case your guess was lower than the number of the computer, the
output should be: \emph{My number is larger!} \\
If you guess was higher than
the number of the computer, the output should read: \\
\emph{My number is smaller!}

It is necessary to seed the random number generator when you start your program
(cf. Section~\ref{Random seeds}.
You could use the {\tt time()} function for this. It returns the actual time measured
in seconds since 1971.

\begin{verbatim}
    srand(time(NULL));   /*Initialisation of the random number generator*/
\end{verbatim}

When you found the right answer, the computer should congratulate you. The 
program should also display the number of tries that where needed in order
to guess the number. 

The program should also keep the a `High-Score', that gets updated once
our number of trials is lower than any previous try.
The High-Score (the number of minimal guesses) should be stored in a
 {\tt struct}, together with the name of the player.

The \texttt{High-Score()} function should ask for your name and store it, when 
your current number of tries is lower than the previous High-Score value. 

The program then gives you the chance to play again or stop the game by pressing
 'q' on the keyboard.



\end{exercise}








\appendix
% LaTeX source for textbook ``How to think like a computer scientist''
% Copyright (C) 2009  Thomas Scheffler


\chapter{Coding Style}
\section{A short guide on style}
\index{style}
\index{coding style}


In the last few sections, I used the phrase ``by convention''
several times to indicate design decisions that are arbitrary
in the sense that there are no significant reasons to do things
one way or another, but dictated by convention.

In these cases, it is to your advantage to be familiar with
convention and use it, since it will make your programs easier
for others to understand.  At the same time, it is important to
distinguish between (at least) three kinds of rules:

\begin{description}

\item[Divine law:]  This is my phrase to indicate a rule that
is true because of some underlying principle of logic or
mathematics, and that is true in any programming language
(or other formal system).  For example, there is no way to
specify the location and size of a bounding box using fewer
than four pieces of information.  Another example is that adding
integers is commutative.  That's part of the definition of
addition and has nothing to do with C.

\item[Rules of C:]  These are the syntactic and semantic
rules of C that you cannot violate, because the
resulting program will not compile or run.  Some are arbitrary;
for example, the fact that the {\tt =} symbol represents
assignment and {\em not} equality.  Others reflect
underlying limitations of the compilation or execution process.
For example, you have to specify the types of parameters, but
not arguments.

\item[Style and convention:]  There are a lot of rules that
are not enforced by the compiler, but that are essential for
writing programs that are correct, that you can debug and
modify, and that others can read.  Examples include indentation
and the placement of squiggly braces, as well as conventions
for naming variables, functions and types.

\end{description}

In this section I will briefly summarize the coding style used within
this book. It follows loosely the "Nasa C Style Guide" 
\footnote{www.scribd.com/doc/6878959/NASA-C-programming-guide}
and its main
intent is on readability rather than saving space or typing effort.

%/ref 
Since C has such a long history of usage, many 
different coding styles have been developed and used. It is important
that you can read them and follow one particular scheme in all
your code. This makes it much more accessible should you find
yourself in a position where you have to share your work with other 
people or have to access code written by your younger self - many years ago...

\section{Naming conventions and capitalization rules}

As a general rule, you should always choose meaningful names for 
your identifiers. Ideally the name of a variable or function already explains
its behaviour or use.

It may be more typing effort to use a function named  {\tt FindSubString()} 
rather than  {\tt FndSStr()}. However, the former is almost self describing 
and might save you a lot in debugging-time.   

\textbf{Don't use single letter variable names!}

Similarly to functions, you should give your variables names that
speak for themselves and make clear what values will be stored
by this variable.
There are few noticeable exceptions to this rule:
People use {\tt i}, {\tt j} and {\tt k} as counter variables in loops and
for spacial coordinates people use {\tt x}, {\tt y} and {\tt z}.
Use these conventions if they suit you. Don't try to invent new
conventions all by yourself.


The following capitalization style shold be used for the different elements in your
program. The consistent use of one style gives the programmer and the reader
of the source code a quick way to determine the meaning of different items
in your program:


\begin{description}
\item[variableNames: ] variable names always start with lower-case, multiple 
words are separated by capitalizing the first letter. 
\item[CONSTANTS: ] use all upper case letters. In order to avoid name space
collisions it might be necessary to use a prefix such as {\tt MY\_CONSTANT}.
\item[FunctionNames:] start always with upper case and should possibly 
contain a verb describing the function. Names for functions that test values should 
start with '{\tt Is}' or '{\tt Are}'.  
\item[UserDefinedTypes\_t:] always end in '{\tt \_t}'. Type names names must be 
capitalised in order to avoid conflict with POSIX names.
\item[pointerNames\_p:] in order to visually separate pointer variables from
ordinary variables you should consider ending pointers with '{\tt \_p}'.
\end{description}

%%
\section{Bracing style}

There exist different bracing or indent styles that serve the goal
to make your code more readable through the use of a consistent 
indentation for control block structures.
The styles differ in the way the braces are indented with the rest
of the control block.
This book uses the BSD/Allman Style because its is the most 
readable of the four. It needs more horizontal space than the K\&R Style
but it makes it very easy to track opening and closing braces.

When you are writing programs, make sure that you are using one
style consistently. In larger projects all contributors should agree
on the style they are using. Modern programming environments like
Eclipse support you through the automatic enforcement of a single style.

\begin{verbatim}
/*Whitesmiths Style*/
   if (condition)            
       {
       statement1; 
       statement2;
       }
\end{verbatim}

Is named after Whitesmiths C, an early commercial C compiler that 
used this style in its examples. Some people refer to it as the 
One True Brace Style.

\begin{verbatim}

/*GNU Style*/
   if (condition)
     {
       statement1;
       statement2;
     }
\end{verbatim}

Indents are always four spaces per level, with the braces halfway between the outer and inner indent levels.

\begin{verbatim}

/*K&R/Kernel Style*/
   if (condition) {
       statement1;
       statement2;
   }
\end{verbatim}

This style is named after the programming examples in the book
\emph{The C Programming Language} by Brian W. Kernighan and 
Dennis Ritchie (the C inventors). 

The K\&R style is the style that is hardest to read. 
The opening brace happens to be at the far right side of the control statement
and can be hard to find. The braces therefore have different indentation levels.
Nevertheless, many C programs use this style. So you should be able 
to read it.

\begin{verbatim}

/*BSD/Allman Style*/
   if (condition)
   {
       statement1;
       statement2;
   }
\end{verbatim}

This style is used for all the examples in this book.

\section{Layout}

Block comments should be used at the top of your file, before all
function declarations, to explain the purpose of the program and give additional
information.

You should also use a similar documentation style before every 
relevant function in your program.
\begin{verbatim}

/*
 * File:     test.c
 * Author:   Peter Programmer
 * Date:     May, 29th, 2009
 *
 * Purpose: to demonstrate good programming
 *          practise
 * /

#include <stdlib.h>

/*
 * main function, does not use arguments
 */

int main (void)
{
    return EXIT_SUCCESS;
}
 
\end{verbatim}




% LaTeX source for textbook ``How to think like a computer scientist''
% Copyright (C) 2009  Thomas Scheffler


\chapter{ASCII-Table}
\label{ASCII-Table}
\vskip -3em

\begin{tabular}{|c|c|c|c||c|c|c|c|}
\hline
Dec & Hex & Oct & Character & Dec & Hex & Oct & Character\\
\hline
0 & 0x00 & 000 & NUL & 32 & 0x20 & 040 & SP\\
1 & 0x01 & 001 & SOH & 33 & 0x21 & 041 & ! \\
2 & 0x02 & 002 & STX & 34 & 0x22 & 042 & "'\\
3 & 0x03 & 003 & ETX & 35 & 0x23 & 043 & \# \\
4 & 0x04 & 004 & EOT & 36 & 0x24 & 044 & \$ \\
5 & 0x05 & 005 & ENQ & 37 & 0x25 & 045 & \% \\
6 & 0x06 & 006 & ACK & 38 & 0x26 & 046 & \& \\
7 & 0x07 & 007 & BEL & 39 & 0x27 & 047 & ' \\
8 & 0x08 & 010 & BS & 40 & 0x28 & 050 & (  \\
9 & 0x09 & 011 & TAB & 41 & 0x29 & 051 &  ) \\
10 & 0x0A & 012 & LF & 42 & 0x2A & 052 & * \\
11 & 0x0B & 013 & VT & 43 & 0x2B & 053 & + \\
12 & 0x0C & 014 & FF & 44 & 0x2C & 054 & , \\
13 & 0x0D & 015 & CR & 45 & 0x2D & 055 & - \\
14 & 0x0E & 016 & SO & 46 & 0x2E & 056 & . \\
15 & 0x0F & 017 & SI & 47 & 0x2F & 057 & / \\
16 & 0x10 & 020 & DLE & 48 & 0x30 & 060 & 0 \\
17 & 0x11 & 021 & DC1 & 49 & 0x31 & 061 & 1 \\
18 & 0x12 & 022 & DC2 & 50 & 0x32 & 062 & 2 \\
19 & 0x13 & 023 & DC3 & 51 & 0x33 & 063 & 3 \\
20 & 0x14 & 024 & DC4 & 52 & 0x34 & 064 & 4 \\
21 & 0x15 & 025 & NAK & 53 & 0x35 & 065 & 5 \\
22 & 0x16 & 026 & SYN & 54 & 0x36 & 066 & 6 \\
23 & 0x17 & 027 & ETB & 55 & 0x37 & 067 & 7 \\
24 & 0x18 & 030 & CAN & 56 & 0x38 & 070 & 8 \\
25 & 0x19 & 031 & EM & 57 & 0x39 & 071 & 9 \\
26 & 0x1A & 032 & SUB & 58 & 0x3A & 072 & : \\
27 & 0x1B & 033 & ESC & 59 & 0x3B & 073 & ; \\
28 & 0x1C & 034 & FS & 60 & 0x3C & 074 & "< \\
29 & 0x1D & 035 & GS & 61 & 0x3D & 075 & =\\
30 & 0x1E & 036 & RS & 62 & 0x3E & 076 & "> \\
31 & 0x1F & 037 & US & 63 & 0x3F & 077 & ? \\
\hline
\end{tabular}
\pagebreak

\begin{longtable}{|c|c|c|c||c|c|c|c|}
\hline
Dec & Hex & Oct & Character & Dec & Hex & Oct & Character\\
\hline
64 & 0x40 & 100 & @ & 96 & 0x60 & 140 & ` \\
65 & 0x41 & 101 & A & 97 & 0x61 & 141 & a \\
66 & 0x42 & 102 & B & 98 & 0x62 & 142 & b \\
67 & 0x43 & 103 & C & 99 & 0x63 & 143 & c \\
68 & 0x44 & 104 & D & 100 & 0x64 & 144 & d \\
69 & 0x45 & 105 & E & 101 & 0x65 & 145 & e \\
70 & 0x46 & 106 & F & 102 & 0x66 & 146 & f \\
71 & 0x47 & 107 & G & 103 & 0x67 & 147 & g \\
72 & 0x48 & 110 & H & 104 & 0x68 & 150 & h \\
73 & 0x49 & 111 & I & 105 & 0x69 & 151 & i \\
74 & 0x4A & 112 & J & 106 & 0x6A & 152 & j \\
75 & 0x4B & 113 & K & 107 & 0x6B & 153 & k \\
76 & 0x4C & 114 & L & 108 & 0x6C & 154 & l \\
77 & 0x4D & 115 & M & 109 & 0x6D & 155 & m \\
78 & 0x4E & 116 & N & 110 & 0x6E & 156 & n \\
79 & 0x4F & 117 & O & 111 & 0x6F & 157 & o \\
80 & 0x50 & 120 & P & 112 & 0x70 & 160 & p \\
81 & 0x51 & 121 & Q & 113 & 0x71 & 161 & q \\
82 & 0x52 & 122 & R & 114 & 0x72 & 162 & r \\
83 & 0x53 & 123 & S & 115 & 0x73 & 163 & s \\
84 & 0x54 & 124 & T & 116 & 0x74 & 164 & t \\
85 & 0x55 & 125 & U & 117 & 0x75 & 165 & u \\
86 & 0x56 & 126 & V & 118 & 0x76 & 166 & v \\
87 & 0x57 & 127 & W & 119 & 0x77 & 167 & w \\
88 & 0x58 & 130 & X & 120 & 0x78 & 170 & x \\
89 & 0x59 & 131 & Y & 121 & 0x79 & 171 & y \\
90 & 0x5A & 132 & Z & 122 & 0x7A & 172 & z \\
91 & 0x5B & 133 & [ & 123 & 0x7B & 173 & \{ \\
92 & 0x5C & 134 & $\backslash$ & 124 & 0x7C & 174 & $\mid$\\
93 & 0x5D & 135 & ] & 125 & 0x7D & 175 & \} \\
94 & 0x5E & 136 & \^{} & 126 & 0x7E & 176 & "~ \\
95 & 0x5F & 137 & \_ & 127 & 0x7F & 177 & DEL \\
\hline
\end{longtable}


\printindex

\end{document}



