% LaTeX source for textbook ``How to think like a computer scientist''
% Copyright (C) 1999  Allen B. Downey
% Copyright (C) 2009  Thomas Scheffler

\setcounter{chapter}{1}
\chapter{Variables and types}

\section{More output}
\index{output}
\index{statement!output}

As I mentioned in the last chapter, you can put as many statements as
you want in {\tt main()}.  For example, to output more than one line:

\begin{verbatim}

  #include <stdio.h>
  #include <stdlib.h>

  /* main: generate some simple output */

  int main (void)
  {
        printf ("Hello World.\n");		    /* output one line */
        printf ("How are you?\n");		    /* output another line */       
        return (EXIT_SUCCESS);
  }

\end{verbatim}
%
As you can see, it is legal to put comments at the
end of a line, as well as on a line by themselves.

\index{String}
\index{type!String}

The phrases that appear in quotation marks are called {\bf strings},
because they are made up of a sequence (string) of letters.  Actually,
strings can contain any combination of letters, numbers, punctuation
marks, and other special characters.

\index{newline}

Often it is useful to display the output from multiple output
statements all on one line.  You can do this by leaving out
the {\tt $\backslash$n} from the first {\tt printf}:

\begin{verbatim}
  int main (void)
  {
        printf ("Goodbye, ");
        printf ("cruel world!\n");	     
        return (EXIT_SUCCESS);
  }
\end{verbatim}
%
In this case the output appears on a single line as
{\tt Goodbye, cruel world!}.  Notice that there is a space
between the word ``Goodbye,'' and the second quotation mark.
This space appears in the output, so it affects the behavior
of the program.

Spaces that appear outside of quotation marks generally do
not affect the behavior of the program.  For example, I
could have written:

\begin{verbatim}
  int main(void)
  {
  printf("Goodbye, ");
  printf("cruel world!\n");	     
  return(EXIT_SUCCESS);
  }
\end{verbatim}
%
This program would compile and run just as well as the original.
The breaks at the ends of lines (newlines) do not affect
the program's behavior either, so I could have written:

\begin{verbatim}
  int main(void){printf("Goodbye, ");printf("cruel world!\n");
  return(EXIT_SUCCESS);}
\end{verbatim}
%
That would work, too, although you have probably noticed that
the program is getting harder and harder to read.  Newlines and
spaces are useful for organizing your program visually, making
it easier to read the program and locate syntax errors.

\section{Values}
\index{value}
\index{type}

Computer programs operate on values stored in 
computer memory.
A value ---like a letter or
a number--- is one of the fundamental things that a program manipulates.  
The only values we have
manipulated so far are the strings we have been outputting, like
{\tt "Hello, world."}.  You (and the compiler) can identify
these string values because they are enclosed in quotation marks.
%constant values

There are different kinds of values, including integers and characters.
It is important for the program to know exactly what kind of value
is manipulated because not all manipulations will make sense on all
values.
We therefore distinguish between different {\bf types} of values.  
%Example 'a' + 'a'

An integer is a whole number like 1 or 17.  You can output
integer values in a similar way as you output strings:

\begin{verbatim}
   printf("%i\n", 17);
\end{verbatim}
%
When we look at the \texttt{printf()} statement more closely, we
notice that the value we are outputting no longer appears
inside the quotes, but behind them separated by comma.
The string is still there, but now contains a {\tt \%i} instead of
any text. 
The {\tt \%i} a placeholder that tells the \texttt{printf()} command
to print an integer value. Several such placeholders exist
for different data types and formatting options of the output.
We will see the next one just now.

A character value is a letter or digit or punctuation mark
enclosed in single quotes, like {\tt 'a'} or {\tt '5'}.
You can output character values in a similar way:

\begin{verbatim}
   printf("%c\n", '}');
\end{verbatim}
%
This example outputs a single closing curly-bracket on a line
by itself. It uses the {\tt \%c} placeholder to signify the output of a character
value.

It is easy to confuse different types of values, like {\tt "5"}, {\tt
'5'} and {\tt 5}, but if you pay attention to the punctuation, it
should be clear that the first is a string, the second is a character
and the third is an integer.  The reason this distinction is important
should become clear soon.

\section {Variables}
\index{variable}
\index{value}

One of the most powerful features of a programming language is the
ability to manipulate values through the use of {\bf variables}.  So far
the values that we have used in our statements where fixed to what 
was written in the statement. Now we will use a variable as a named 
location that stores a value.  

Just as there are different types of values (integer, character,
etc.), there are different types of variables.  When you create a new
variable, you have to declare what type it is.  For example, the
character type in C is called {\tt char}.  The following statement
creates a new variable named {\tt fred} that has type {\tt char}.

\begin{verbatim}
    char fred;
\end{verbatim}
%
This kind of statement is called a {\bf declaration}.

The type of a variable determines what kind of values it can
store.  A {\tt char} variable can contain characters, and it should
come as no surprise that {\tt int} variables can store integers.

Contrary to other programming languages, C does not have a 
dedicated variable type for the storage of string values. We will see in
a later chapter how string values are stored in C. 
%but we
%are going to skip that for now (see Chapter~\ref{strings}).

\index{declaration}
\index{statement!declaration}

To create an integer variable, the syntax is 

\begin{verbatim}
    int bob;
\end{verbatim}
%
where {\tt bob} is the arbitrary name you choose to identify the
variable.  In general, you will want to make up variable names
that indicate what you plan to do with the variable.  For
example, if you saw these variable declarations:

\begin{verbatim}
    char first_letter;
    char last_letter;
    int hour, minute;
\end{verbatim}
%
you could probably make a good guess at what values
would be stored in them.  This example
also demonstrates the syntax for declaring multiple variables
with the same type: {\tt hour} and {\tt minute}
are both integers ({\tt int} type).

ATTENTION: The older C89 standard allows variable declarations
only at the beginning of a block of code. It is therefore necessary
to put variable declarations before any other statements,
even if the variable itself is only needed much later in your program.

\section{Assignment}
\index{assignment}
\index{statement!assignment}

Now that we have created some variables, we would like to
store values in them.  We do that with an {\bf assignment
statement}.

\begin{verbatim}
    first_letter = 'a';   /* give first_letter the value 'a' */
    hour = 11;            /* assign the value 11 to hour */
    minute = 59;          /* set minute to 59 */
\end{verbatim}
%
This example shows three assignments, and the comments show
three different ways people sometimes talk about assignment
statements.  The vocabulary can be confusing here, but the
idea is straightforward:

\begin{itemize}

\item When you declare a variable, you create a named storage location.

\item When you make an assignment to a variable, you give it a value.

\end{itemize}

A common way to represent variables on paper is to draw a box
with the name of the variable on the outside and the value
of the variable on the inside.  This kind of figure is called
a {\bf state diagram} because is shows what state each 
variable is in (you can think of it as the variable's ``state of
mind'').
This diagram shows the effect of the three assignment statements:

%\vspace{0.1in}
%\centerline{\epsfig{figure=figs/assign.eps}}
%\vspace{0.1in}

\setlength{\unitlength}{1mm}
\begin{picture}(20,17)
\put(7,12){\large \texttt{first\_letter}}
\put(46,12){\large \texttt{hour}}
\put(74,12){\large \texttt{minute}}
\put(10,0){\framebox(20,10){{\large \textsf{a}}}}
\put(40,0){\framebox(20,10){{\large \textsf{11}}}}
\put(70,0){\framebox(20,10){{\large \textsf{59}}}}
\end{picture}

When we assign values to variables, we have to make sure that
the assigned value correspondents to the type of the variable.
In C  a variable has to have the same type as the
value you assign.  For example, you cannot store a string in
an {\tt int} variable.  The following statement generates a compiler
warning:

\begin{verbatim}
    int hour;
    hour = "Hello.";       /* WRONG !! */
\end{verbatim}
%
This rule is sometimes a source of confusion, because there are many
ways that you can convert values from one type to another, and C
sometimes converts things automatically.  But for now you should
remember that as a general rule variables and values have the same
type, and we'll talk about special cases later.

Another source of confusion is that some strings {\em look}
like integers, but they are not.  For example,
the string {\tt "123"}, which is made up of the
characters {\tt 1}, {\tt 2} and {\tt 3}, is not
the same thing as the {\em number} {\tt 123}.
This assignment is illegal:

\begin{verbatim}
    minute = "59";         /* WRONG!! */
\end{verbatim}
%
\section{Outputting variables}
\label{output variables}

You can output the value of a variable using the same commands
we used to output simple values.

\begin{verbatim}

    int hour, minute;
    char colon;

    hour = 11;
    minute = 59;
    colon = ':';

    printf ("The current time is ");
    printf ("%i", hour);
    printf ("%c", colon);
    printf ("%i", minute);
    printf ("\n"); 
  
\end{verbatim}
%
This program creates two integer variables named {\tt hour} and {\tt
minute}, and a character variable named {\tt colon}.  It assigns
appropriate values to each of the variables and then uses a series
of output statements to generate the following:

\begin{verbatim}
    The current time is 11:59
\end{verbatim}

When we talk about ``outputting a variable,'' we mean outputting the
{\em value} of the variable.  The name of a variable only has significance for
the programmer. The compiled program no longer contains a human readable
reference to the variable name in your program. 

The \texttt{ printf()} command is capable of outputting several variables
in a single statement. To do this, we need to put placeholders
in the so called \emph{format string}, that indicate the position where the variable value will
be put. The variables will be inserted in the order of their appearance in 
the statement. It is important to observe the right order and type for the variables.

By using a single output statement, we can make the previous program more
concise:

\begin{verbatim}
    
    int hour, minute;
    char colon;

    hour = 11;
    minute = 59;
    colon = ':';

    printf ("The current time is %i%c%i\n", hour, colon, minute);
    
\end{verbatim}
%
On one line, this program outputs a string, two integers and a character.  Very impressive!

\section{Keywords}
\index{keyword}

A few sections ago, I said that you can make up any name you
want for your variables, but that's not quite true.  There
are certain words that are reserved in C because they are
used by the compiler to parse the structure of your program,
and if you use them as variable names, it will get confused.
These words, called {\bf keywords}, include {\tt int},
{\tt char}, {\tt void} and many more.

\vskip 1em

\setlength{\fboxsep}{6pt} 
\begin{center}
\begin{boxedminipage}[c]{.9\linewidth}
\begin{center}
\begin{multicols}{5}[\underline{Reserved keywords in the C language}]
\begin{verbatim}
auto 
break 
case 
char 
const 
continue 
default 
do 
double 
else 
enum 
extern 
float 
for 
goto 
if 
inline 
int 
long 
register 
restrict 
return 
short 
signed 
sizeof 
static 
struct 
switch 
typedef 
union 
unsigned 
void 
volatile 
while 
_Bool 
_Complex 
_Imaginary 
\end{verbatim}
\end{multicols}
\end{center}
\end{boxedminipage}
\end{center}

\vskip 1em
The complete list of keywords is included in the C Standard, which
is the official language definition adopted by the the International
Organization for Standardization (ISO) on September 1, 1998.  

%You can download a copy electronically from
%
%\begin{verbatim}
%    http://www.ansi.org/
%\end{verbatim}
%
Rather than memorize the list, I would suggest that you
take advantage of a feature provided in many development
environments: code highlighting.  As you type, different
parts of your program should appear in different colors.  For
example, keywords might be blue, strings red, and other code
black.  If you type a variable name and it turns blue, watch
out!  You might get some strange behavior from the compiler.

\section{Operators}
\label{operators}
\index{operator}


{\bf Operators} are special symbols that are used to represent
simple computations like addition and multiplication.  Most
of the operators in C do exactly what you would expect them
to do, because they are common mathematical symbols.  For
example, the operator for adding two integers is {\tt +}.

The following are all legal C expressions whose meaning is
more or less obvious:

\begin{verbatim}
    1+1        hour-1       hour*60+minute     minute/60
\end{verbatim}
%
{\bf Expressions} can contain both variables
names and values.  In each case the name of the variable is
replaced with its value before the computation is performed.

\index{expression}

Addition, subtraction and multiplication all do what you
expect, but you might be surprised by division.  For example,
the following program:

\begin{verbatim}

  int hour, minute;
  hour = 11;
  minute = 59;
  printf ("Number of minutes since midnight: %i\n", hour*60 + minute);
  printf ("Fraction of the hour that has passed: %i\n", minute/60);

\end{verbatim}
%
would generate the following output:

\begin{verbatim}
    Number of minutes since midnight: 719
    Fraction of the hour that has passed: 0
\end{verbatim}
%
The first line is what we expected, but the second line is
odd.  The value of the variable {\tt minute} is 59, and
59 divided by 60 is 0.98333, not 0.  The reason for the
discrepancy is that C is performing {\bf integer division}.

\index{type!int}
\index{integer division}
\index{arithmetic!integer}
\index{division!integer}
\index{operand}

When both of the {\bf operands} are integers (operands are the things
operators operate on), the result must also be an integer,
and by definition integer division always rounds {\em down},
even in cases like this where the next integer is so close.

A possible alternative in this case is to calculate a percentage
rather than a fraction:

\begin{verbatim}
    printf ("Percentage of the hour that has passed: ");
    printf ("%i\n", minute*100/60);
\end{verbatim}
%
The result is:

\begin{verbatim}
    Percentage of the hour that has passed: 98
\end{verbatim}
%
Again the result is rounded down, but at least now the answer
is approximately correct.  In order to get an even more accurate
answer, we could use a different type of variable, called
floating-point, that is capable of storing fractional values.
We'll get to that in the next chapter.

\section{Order of operations}
\index{precedence}
\index{order of operations}

When more than one operator appears in an expression the order
of evaluation depends on the rules of {\bf precedence}.  A
complete explanation of precedence can get complicated, but
just to get you started:

\begin{itemize}

\item Multiplication and division happen before
addition and subtraction.  So {\tt 2*3-1} yields 5, not 4, and {\tt
2/3-1} yields -1, not 1.
%(remember that in integer division {\tt 2/3} is 0).

\item If the operators have the same precedence they are evaluated
from left to right.  So in the expression {\tt minute*100/60},
the multiplication happens first, yielding {\tt 5900/60}, which
in turn yields {\tt 98}.  If the operations had gone from right
to left, the result would be {\tt 59*1} which is {\tt 59}, which
is wrong.

\item Any time you want to override the rules of precedence (or
you are not sure what they are) you can use parentheses.  Expressions
in parentheses are evaluated first, so {\tt 2*(3-1)} is 4.
You can also use parentheses to make an expression easier to
read, as in {\tt (minute*100)/60}, even though it doesn't
change the result.

\end{itemize}

\section{Operators for characters}
\index{character operator}
\index{operator!character}

Interestingly, the same mathematical operations that work on
integers also work on characters.  For example,

\begin{verbatim}
    char letter;
    letter = 'a' + 1;
    printf ("%c\n", letter);
\end{verbatim}
%
outputs the letter {\tt b}.  Although it is syntactically legal
to multiply characters, it is almost never useful to do it.

Earlier I said that you can only assign integer values to
integer variables and character values to character variables,
but that is not completely true.  In some cases, C converts
automatically between types.  For example, the following is
legal.

\begin{verbatim}
    int number;
    number = 'a';
    printf ("%i\n", number);
\end{verbatim}
%
The result is 97, which is the number that is used internally
by C to represent the letter {\tt 'a'}.  However, it is
generally a good idea to treat characters as characters, and
integers as integers, and only convert from one to the other
if there is a good reason.

Automatic type conversion is an example of a common problem in designing a
programming language, which is that there is a conflict between {\bf
formalism}, which is the requirement that formal languages should have
simple rules with few exceptions, and {\bf convenience}, which is the
requirement that programming languages be easy to use in practice.

More often than not, convenience wins, which is usually good for
expert programmers, who are spared from rigorous but unwieldy
formalism, but bad for beginning programmers, who are often baffled
by the complexity of the rules and the number of exceptions.  In this
book I have tried to simplify things by emphasizing the rules and
omitting many of the exceptions.


\section{Composition}
\index{composition}
\index{expression}

So far we have looked at the elements of a programming
language---variables, expressions, and statements---in
isolation, without talking about how to combine them.

One of the most useful features of programming languages
is their ability to take small building blocks and
{\bf compose} them.  For example, we know how to multiply
integers and we know how to output values; it turns out we can
do both at the same time:

\begin{verbatim}
    printf ("%i\n", 17 * 3);
\end{verbatim}
%
Actually, I shouldn't say ``at the same time,'' since in reality
the multiplication has to happen before the output, but
the point is that any expression, involving numbers, characters,
and variables, can be used inside an output statement.  We've
already seen one example:

\begin{verbatim}
    printf ("%i\n", hour * 60 + minute);
\end{verbatim}
%
You can also put arbitrary expressions on the right-hand
side of an assignment statement:

\begin{verbatim}
    int percentage;
    percentage = (minute * 100) / 60;
\end{verbatim}
%
This ability may not seem so impressive now, but we will see
other examples where composition makes it possible
to express complex computations neatly and concisely.

WARNING: There are limits on where you can use certain
expressions; most notably, the left-hand side of an assignment
statement has to be a {\em variable} name, not an expression.
That's because the left side indicates the storage location
where the result will go.  Expressions
do not represent storage locations, only values.  So the
following is illegal:  {\tt minute + 1 = hour;}.

\section{Glossary}

\begin{description}

\item[variable:] A named storage location for values.  All
variables have a type, which determines which values it can
store.

\item[value:] A letter, or number, or other thing that can be
stored in a variable.  

\item[type:] The meaning of values.  The types
we have seen so far are integers ({\tt int} in C) and characters ({\tt
char} in C).

\item[keyword:]  A reserved word that is used by the compiler
to parse programs.  Examples we have seen include {\tt int},
{\tt void} and {\tt char}.

\item[statement:] A line of code that represents a command or
action.  So far, the statements we have seen are declarations,
assignments, and output statements.

\item[declaration:] A statement that creates a new variable and
determines its type.

\item[assignment:] A statement that assigns a value to a variable.

\item[expression:] A combination of variables, operators and
values that represents a single result value.  Expressions also
have types, as determined by their operators and operands.

\item[operator:] A special symbol that represents a simple
computation like addition or multiplication.

\item[operand:] One of the values on which an operator operates. 

\item[precedence:] The order in which operations are evaluated.

\item[composition:] The ability to combine simple
expressions and statements into compound statements and expressions
in order to represent complex computations concisely.

\index{variable}
\index{value}
\index{type}
\index{keyword}
\index{statement}
\index{assignment}
\index{expression}
\index{operator}
\index{operand}
\index{composition}

\end{description}


\section{Exercises}
\setcounter{exercisenum}{0}

% LaTeX source for textbook ``How to think like a computer scientist''
% Copyright (C) 1999  Allen B. Downey
% Copyright (C) 2009  Thomas Scheffler

\begin{exercise}
\label{ex.date}

\begin{enumerate}

\item Create a new program named {\tt MyDate.c}.  Copy or
type in something like the "Hello, World" program and make
sure you can compile and run it.

\item Following the example in Section~\ref{output variables}, write a program
that creates variables named {\tt day}, {\tt month}
and {\tt year}
What type is each variable?

Assign values to those variables that represent today's date.

\item Print the value of each variable on a line by itself.  This is
an intermediate step that is useful for checking that everything is
working so far.

\item Modify the program so that it prints the date in standard
American form: {\tt {\tt mm/dd/yyyy}}.

\item Modify the program again so that the total output is:

\begin{verbatim}
American format:
3/18/2009
European format:
18.3.2009
\end{verbatim}

\end{enumerate}

The point of this exercise is to use the output function {\tt printf} to display
values with different types, and to
practice developing programs gradually by adding a few statements
at a time.



\end{exercise}


\begin{exercise}

\begin{enumerate}

\item Create a new program called {\tt MyTime.c}.  From now
on, I won't remind you to start with a small, working program,
but you should.

\item Following the example in Section~\ref{operators}, create variables
named {\tt hour}, {\tt minute} and {\tt second}, and assign
them values that are roughly the current time.  Use a 24-hour
clock, so that at 2pm the value of {\tt hour} is 14.

\item Make the program calculate and print the number of
seconds since midnight.

\item Make the program calculate and print the number of
seconds remaining in the day.

\item Make the program calculate and print the percentage of
the day that has passed.

\item Change the values of {\tt hour}, {\tt minute} and {\tt second}
to reflect the current time (I assume that some time has elapsed), and
check to make sure that the program works correctly with different
values.

\end{enumerate}

The point of this exercise is to use some of the arithmetic
operations, and to start thinking about compound entities like the
time of day that are represented with multiple values.  Also,
you might run into problems computing percentages with {\tt ints},
which is the motivation for floating point numbers in the next
chapter.

HINT: you may want to use additional variables to hold values
temporarily during the computation.  Variables like this, that
are used in a computation but never printed, are sometimes called
intermediate or temporary variables.

\end{exercise}






