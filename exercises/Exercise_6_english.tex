% LaTeX source for textbook ``How to think like a computer scientist''
% Copyright (C) 1999  Allen B. Downey
% Copyright (C) 2009  Thomas Scheffler

%%%%%%%%%%%%%%%%%%%%%%%%%%%%%%%%%%%%%%

\begin{exercise}\label{infloop}
%changed the condition on the loop so that it will terminate
%(was this *supposed* to be an infinite loop?)
\begin{verbatim}
    void Loop(int n) 
    {
        int i = n;
        while (i > 1) 
        {
            printf ("%i\n",i);
            if (i%2 == 0) 
            {
                i = i/2;
            } 
            else 
            {
                i = i+1;
            }
        }
    }

    int main (void) 
    {
        Loop(10);
        return EXIT_SUCCESS;
    }
\end{verbatim}
%
\begin{enumerate}

\item  Draw a table that shows the value of the variables {\tt i} and {\tt n} during the execution of the program. 
The table should contain one column for each variable and one line for each iteration.


\item What is the output of this program?

\end{enumerate}
\end{exercise}

%%%%%%%%%%%%%%%%%%%%%%%%%%%%%%%%%%%%%%



\begin{exercise}
In Exercise~\ref{ex.power} we wrote a recursive version of {\tt
Power()}, which takes a double {\tt x} and an integer {\tt n} and
returns $x^n$.  Now write an iterative function to perform the same
calculation.
\end{exercise}

%%%%%%%%%%%%%%%%%%%%%%%%%%%%%%%%%%%%%%

\begin{exercise}
Let's say you are given a number, $a$, and you want to find
its square root.  One way to do that is to start with a very
rough guess about the answer, $x_0$, and then improve
the guess using the following formula:

\begin{displaymath}
x_1 = (x_0 + a/x_0) / 2
\end{displaymath}

For example, if we want to find the square root of 9, and
we start with $x_0 = 6$, then $x_1 = (6 + 9/6) /2 = 15/4 = 3.75$,
which is closer.

We can repeat the procedure, using $x_1$ to calculate $x_2$,
and so on.  In this case, $x_2 = 3.075$ and $x_3 = 3.00091$.
So that is converging very quickly on the right answer (which
is 3).

Write a function called {\tt SquareRoot} that takes a {\tt double}
as a parameter and that returns an approximation of the square
root of the parameter, using this algorithm.  You may not use
the {\tt sqrt()} function from the {\tt math.h} library.

As your initial guess, you should use $a/2$.  Your function should
iterate until it gets two consecutive estimates that differ by
less than 0.0001; in other words, until the absolute value of
$x_n - x_{n-1}$ is less than 0.0001.  You can use the built-in
{\tt fabs()} function from the {\tt math.h} library to calculate the absolute value.
\end{exercise}

%%%%%%%%%%%%%%%%%%%%%%%%%%%%%%%%%%%%%%


\begin{exercise}
One way to evaluate $e^{-x^2}$ is to use the infinite series expansion

\begin{displaymath}
e^{-x^2} = 1 - 2x + 3x^2/2! - 4x^3/3! + 5x^4/4! - ...
\end{displaymath}

In other words, we need to add up a series of terms where the $i$th
term is equal to $(-1)^i(i+1) x^i / i!$.  Write a function named {\tt Gauss()}
that takes {\tt x} and {\tt n} as arguments and that returns the sum
of the first {\tt n} terms of the series.  You should not use {\tt
factorial()} or {\tt pow()}.

\end{exercise}

%%%%%%%%%%%%%%%%%%%%%%%%%%%%%%%%%%%%%%

